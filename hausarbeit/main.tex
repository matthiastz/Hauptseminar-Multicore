\documentclass[a4paper,11pt]{scrartcl}
\usepackage[T1]{fontenc}
\usepackage[utf8]{inputenc}
\usepackage{lmodern}
\usepackage{ngerman}
\usepackage{graphicx}

% \title{OpenTuner: An Extensible Framework for Program Autotuning
% \includegraphics{bilder/tuc-logo-black.pdf}
% 
  
\title{\includegraphics[width=0.6\textwidth]{bilder/tuc-logo-black.pdf}
    OpenTuner:~An~Extensible~Framework\\for~Program~Autotuning
}
\author{Seminararbeit\\Autor: Matthias Tietz}
\date{\today}

%%%%%%%%%%%%%%%%%%%%%%%%%%%%%%%%%%%%%%%%%%%%%%%%%%%%%%%%%%%%%%%%%%%%%%%%%
%%%%%%%%%%%%%%%%%%%%%%%%%%%%%%%%%%%%%%%%%%%%%%%%%%%%%%%%%%%%%%%%%%%%%%%%%

\begin{document}


\maketitle \thispagestyle{empty} \newpage

%%% Informationen/Leerseite %%%
\thispagestyle{empty}
~
\vfill
\noindent % Einrückung vermeiden
Technische Universität Chemnitz\\
Fakultät für Informatik\\
Professur Praktische Informatik\\
Hauptseminar Multicore-Programmierung\\
Wintersemester 2016/2017\\

\noindent
OpenTuner: An Extensible Framework for Program Autotuning\\
Autor: Matthias Tietz\\
Bachelor Informatik, 5.~Fachsemester

\newpage
\tableofcontents \newpage

%%%%%%%%
%% TODO: Kopfzeile Information der aktuellen Section ??
%%%%%%%%

%%%%%%%%
%% TODO: Sichtwortverzeichnis? : Autotuner, Param, Konfig, ...
%%%%%%%%

\section{Einleitung}
% sub: Related Work, Begriffe

Programm-Autotuning findet zunehmend Anwendung in Domänen wie Hochleistungsrechnen oder
digitaler Bild-~und~Signalverarbeitung zur Optimierung der entsprechenden Anwendung.
Durch die Verwendung eines Autotuners besteht die Möglichkeit, die Suche nach der
bestmöglichen Programm-Implementierung zu automatisieren. Anstatt ein Programm direkt
zu optimieren, beschreibt der Nutzer eine Menge möglicher Implementierungen, die 
systematisch von geeigneten Techniken durchsucht wird. Das Optimierungsverfahren
mittels Autotuner ist zumeist effizienter als die Optimierung von Hand, da deutlich größere
Suchräume verarbeitet werden können.




\section{Das Framework}
abc

\section{Verwendung}
abc
\subsection{Beispielhafte Anwendungen}

\subsection{Ausprobieren der Techniken}

\section{Fazit}
abc

\section{Literaturverzeichnis}
abc

\section{Eigenständigkeitserklärung}
abc

\end{document}
