\documentclass[a4paper,11pt]{scrartcl}
\usepackage[T1]{fontenc}
\usepackage[utf8]{inputenc}
\usepackage{lmodern}
\usepackage{ngerman}
\usepackage{graphicx}
\usepackage{xspace}

\newcommand{\zB}{\mbox{z.\,B.}\xspace}
\newcommand{\iAllg}{\mbox{i.\,Allg.}\xspace}
\setlength{\parindent}{0em} % Einrückung verhindern

%%%%%%%%%%%%%%%%%%%%%%%%%%%%%%%%%%%%%%%%%%%%%%%%%%%%%%%%%%%%%%%%%%%%%%%%%
%%%%%%%%%%%%%%%%%%%%%%%%%%%%%%%%%%%%%%%%%%%%%%%%%%%%%%%%%%%%%%%%%%%%%%%%%
  
\title{\includegraphics[width=0.6\textwidth]{bilder/tuc-logo-black.pdf}
    OpenTuner:~An~Extensible~Framework\\for~Program~Autotuning
}
\author{Seminararbeit\\Autor: Matthias Tietz}
\date{\today}

%%%%%%%%%%%%%%%%%%%%%%%%%%%%%%%%%%%%%%%%%%%%%%%%%%%%%%%%%%%%%%%%%%%%%%%%%
%%%%%%%%%%%%%%%%%%%%%%%%%%%%%%%%%%%%%%%%%%%%%%%%%%%%%%%%%%%%%%%%%%%%%%%%%

\begin{document}

\maketitle \thispagestyle{empty} \newpage

%%% Informationen/Leerseite %%%
\thispagestyle{empty}
~
\vfill
Technische Universität Chemnitz\\
Fakultät für Informatik\\
Professur Praktische Informatik\\
Hauptseminar Multicore-Programmierung\\
Wintersemester 2016/2017\\

OpenTuner: An Extensible Framework for Program Autotuning\\
Autor: Matthias Tietz\\
Matrikelnummer:~375681\\
Bachelor Informatik, 5.~Fachsemester

\newpage
\tableofcontents \newpage

%%%%%%%%
%% TODO: Kopfzeile Information der aktuellen Section ??
%%%%%%%%

%%%%%%%%
%% TODO: Sichtwortverzeichnis? : Autotuner, Param, Konfig, ...
%%%%%%%%

\section{Einleitung}
% sub: Related Work, Begriffe

Programm-Autotuning findet zunehmend Anwendung in Domänen wie Hochleistungsrechnen oder
digitaler Bild-~und~Signalverarbeitung zur Optimierung der entsprechenden Anwendung.
Durch die Verwendung eines Autotuners besteht die Möglichkeit, die Suche nach der
bestmöglichen Programm-Implementierung zu automatisieren. Anstatt ein Programm direkt
zu optimieren, beschreibt der Nutzer eine Menge möglicher Implementierungen, die 
systematisch von geeigneten Techniken durchsucht wird. Das Optimierungsverfahren
mittels Autotuner ist zumeist effizienter als die Optimierung von Hand, da deutlich größere
Suchräume verarbeitet werden können. \newline

Bei der Erstellung eines Autotuners steht oftmals die Verbesserung der Laufzeit
für das spezifische Programm im Vordergrund. OpenTuner bietet neben \texttt{time}\footnote{Zeit, Ausführungszeit}  
alternativ weitere Optimierungsziele, \zB \texttt{accuracy}\footnote{Genauigkeit, \zB Genauigkeit einer Berechnung}  
und \texttt{energy}\footnote{Energie, Energiebedarf eines bestimmten Systems}.
Es ist außerdem möglich, mehrere Ziele bei der Optimierung zu verfolgen. 
So kann man bspw. das Optimierungsziel \texttt{time-accuracy} definieren, also die Ausführungszeit
des Programms verbessern und gleichzeitig die Genauigkeit berücksichtigen. \newline

Bestehende Autotuning-Frameworks wurden \iAllg zielgerichtet für den Einsatz in einer 
bestimmten Domäne entwickelt. \textsc{ATLAS} \cite{atlas} ist ein Projekt aus dem Bereich der Linearen Algebra,
\textsc{FFTW} \cite{fftw} verwendet Autotuning zum Lösen schneller Fourier-Transformationen.
Im Gegensatz dazu verfolgt OpenTuner den Ansatz, ein generelles System zur Erstellung 
von Autotunern für verschiedene Domänen einzuführen. Diese Flexibilität wird grundsätzlich durch eine stark
ausgeprägte Erweiterbarkeit und einen großen Funktionsumfang des Frameworks ermöglicht.
Das durch OpenTuner realisierte Optimierungsverfahren kann auf andere Rechnersysteme übertragen
und mit den dort existierenden Bedingungen wiederholt werden. \newline

Bei der Entwicklung eines Autotuning-Frameworks existieren drei grundsätzliche Herausforderungen.
Zunächst die Möglichkeit für einen Problemfall eine \emph{passende Konfigurations-Repräsentation} wählen
zu können. Das umfasst die Darstellung der notwendigen Datenstrukturen und Bedingungen. Eine gute 
Repräsentation ist entscheidend für die Effizienz des Autotuners. Ebenso kritisch kann die 
\emph{Größe des gültigen Suchraumes} sein, denn durch Kombination verschiedener Parameter sind 
schnell rießige Suchräume möglich. Eine vollständige Suche würde dort oftmals nicht abschließen,
daher bedarf es intelligenter Suchtechniken, welche nur einen kleinen Teil der Konfigurationsmenge
durchsuchen, um ein gutes Ergebnis zu erzielen. Die \emph{Beschaffenheit des Suchraumes} stellt
ebenfalls eine Herausforderung dar, da in vielen praktischen Anwendungen die Suchräume meist ein
hohes Maß an Komplexität besitzen.

%%% ein wenig kürzen %%%

\section{OpenTuner Framework}
Die Architektur des Frameworks ist vergleichsweise kompakt und lässt sich in die folgenden drei 
Komponenten gliedern: Suche, Messung und Ergebnis-Datenbank.


\begin{figure}[h]
\begin{center}
\includegraphics{bilder/smdb}
\cite{OT-paper} \caption{Komponenten des OpenTuner Framework} 
\end{center}   
\end{figure}

Die Menge der Suchtechniken verwendet den Konfigurations-Manipulator, um Konfigurationen lesen und 
schreiben zu können. Ausgewählte Konfigurationen werden durch eine benutzerdefinierte Messfunktion
ausgeführt und anschließend ausgewertet. Die Datenbank dient dem Festhalten der Ergebnisse des Tuning-Vorgangs
und dem Informationsaustausch zwischen Suche und Messung.

%%% ggf. noch Messungen parallel??? %%%

\subsection{Suchtechniken}
OpenTuner stellt im Auslieferungszustand grundlegende Suchtechniken bereit, die auf viele Arten von 
Suchräumen anwendbar sind. Die Verwendung der Suchtechniken findet hierarchisch statt --
die Suchsteuerung spricht eine \texttt{root technique}\footnote{zentrale Technik (sog.~Meta-Technik),
ist den normalen Suchtechniken/Heuristiken übergeordnet} an, die dann während des Tuning-Vorgangs 
Tests auf eine Menge von \texttt{sub-techniques}\footnote{die Standard-Suchtechniken} verteilt. \newline

Die \emph{Ensembles} sind ein Konzept von OpenTuner, welches die Kombination mehrerer Suchtechniken
ermöglicht. Einzelne Suchtechniken tauschen Ergebnisse über die Datenbank aus -- findet eine 
Suchtechnik eine gute Konfiguration, so können andere Techniken davon profitieren.
Suchtechniken bekommen ihre Testanteile dynamisch zugewiesen, abhängig davon wie erfolgreich eine
jeweils ist.

\subsection{Konfigurations-Manipulator}



\section{Verwendung}
abc
\subsection{Beispielhafte Anwendungen}

\subsection{Ausprobieren der Techniken}

\section{Fazit}
abc

\section{Literaturverzeichnis}
abc

\begin{thebibliography}{9}
   
  %%%%%%%%%%%%%%%%%%%
  %%%%%%%%%%%%%%%%%%%
  %  TODO: Zitatstile -> Reihenfolge: Autor, Titel, Jahr, Verlag, Seite, …
  %%%%%%%%%%%%%%%%%%%
  %%%%%%%%%%%%%%%%%%%
  
    \bibitem{OT-paper} Jason Ansel, Shoaib Kamil, Kalyan Veeramachaneni, Jonathan Ragan-Kelley, Jeffrey Bosboom, Una-May O'Reilly, Saman Amarasinghe. \emph{OpenTuner: An Extensible Framework for Program Autotuning.}
    2014.
   
\bibitem{atlas} Liste der Autoren \emph{Titel oder Quelle des zitierten Werkes} ggf. Seiten.
    Jahr. und weiteres.
\bibitem{fftw} Liste der Autoren \emph{Titel oder Quelle des zitierten Werkes} ggf. Seiten.
    Jahr. und weiteres.

\end{thebibliography}


% \section{Eigenständigkeitserklärung}
% abc

\end{document}
