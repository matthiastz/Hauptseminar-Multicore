\documentclass[a4paper,11pt]{scrartcl}
\usepackage[T1]{fontenc}
\usepackage[utf8]{inputenc}
\usepackage{lmodern}
\usepackage{ngerman}
\usepackage{graphicx}
\usepackage{xspace}
\usepackage{listings}
\usepackage{color}
\usepackage[hyphens]{url}
\usepackage{hyperref}

\definecolor{mygreen}{rgb}{0,0.6,0}
\definecolor{mygray}{rgb}{0.5,0.5,0.5}
\definecolor{mymauve}{rgb}{0.58,0,0.82}

% \fontfamily{pzc}\selectfont
% \ttfamily

\lstset{ %
  backgroundcolor=\color{white},   % choose the background color; you must add \usepackage{color} or \usepackage{xcolor}; should come as last argument
  basicstyle=\footnotesize\fontfamily{bch}\selectfont,        
  % the size & type of the fonts that are used for the code
  breakatwhitespace=false,         % sets if automatic breaks should only happen at whitespace
  breaklines=true,                 % sets automatic line breaking
  captionpos=b,                    % sets the caption-position to bottom
  commentstyle=\color{mygreen},    % comment style
  deletekeywords={min,max},            % if you want to delete keywords from the given language
  escapeinside={\%*}{*)},          % if you want to add LaTeX within your code
  extendedchars=true,              % lets you use non-ASCII characters; for 8-bits encodings only, does not work with UTF-8
  frame=single,	                   % adds a frame around the code
  keepspaces=true,                 % keeps spaces in text, useful for keeping indentation of code (possibly needs columns=flexible)
  keywordstyle=\color{blue},       % keyword style
  language=Python,                 % the language of the code
  morekeywords={},                 % if you want to add more keywords to the set
  numbers=none,                    % where to put the line-numbers; possible values are (none, left, right)
  numbersep=5pt,                   % how far the line-numbers are from the code
  numberstyle=\tiny\color{mygray}, % the style that is used for the line-numbers
  rulecolor=\color{black},         % if not set, the frame-color may be changed on line-breaks within not-black text (e.g. comments (green here))
  showspaces=false,                % show spaces everywhere adding particular underscores; it overrides 'showstringspaces'
  showstringspaces=false,          % underline spaces within strings only
  showtabs=false,                  % show tabs within strings adding particular underscores
  stepnumber=2,                    % the step between two line-numbers. If it's 1, each line will be numbered
  stringstyle=\color{mymauve},     % string literal style
  tabsize=2,	                   % sets default tabsize to 2 spaces
%   title=\lstname                   % show the filename of files included with \lstinputlisting; also try caption instead of title
}


% Add new keywords here: ConfigManip, Parameters…
\lstset{emph={%  
    IntegerParameter, ConfigurationManipulator,EnumParameter,format,call_program%
    },emphstyle={\textbf}%
}%


\newcommand{\zB}{\mbox{z.\,B.}\xspace}
\newcommand{\bspw}{\mbox{bspw.}\xspace}
\newcommand{\bzw}{\mbox{bzw.}\xspace}
\newcommand{\iAllg}{\mbox{i.\,Allg.}\xspace}
\newcommand{\ua}{\mbox{u.\,a.}\xspace}
\newcommand{\vs}{\mbox{vs.}\xspace}

\def\CC{{C\nolinebreak[4]\hspace{-.05em}\raisebox{.4ex}{\tiny\bf ++}}}
\setlength{\parindent}{0em} % Einrückung verhindern

%%%%%%%%%%%%%%%%%%%%%%%%%%%%%%%%%%%%%%%%%%%%%%%%%%%%%%%%%%%%%%%%%%%%%%%%%
%%%%%%%%%%%%%%%%%%%%%%%%%%%%%%%%%%%%%%%%%%%%%%%%%%%%%%%%%%%%%%%%%%%%%%%%%
  
\title{\includegraphics[width=0.6\textwidth]{bilder/tuc-logo-black.pdf}
    OpenTuner:~An~Extensible~Framework\\for~Program~Autotuning
}
\author{Seminararbeit\\Autor: Matthias Tietz}
\date{\today}

%%%%%%%%%%%%%%%%%%%%%%%%%%%%%%%%%%%%%%%%%%%%%%%%%%%%%%%%%%%%%%%%%%%%%%%%%
%%%%%%%%%%%%%%%%%%%%%%%%%%%%%%%%%%%%%%%%%%%%%%%%%%%%%%%%%%%%%%%%%%%%%%%%%

\begin{document}

\maketitle \thispagestyle{empty} \newpage

%%% Informationen/Leerseite %%%
\thispagestyle{empty}
~
\vfill
Technische Universität Chemnitz\\
Fakultät für Informatik\\
Professur Praktische Informatik\\
Hauptseminar Multicore-Programmierung\\
Wintersemester 2016/2017\\

OpenTuner: An Extensible Framework for Program Autotuning\\
Autor: Matthias Tietz\\
Matrikelnummer:~375681\\
Bachelor Informatik, 5.~Fachsemester

\newpage
\tableofcontents \newpage


%%% section %%%%%%%%%%%%%%%%%%%%%%%%%%%%%%%%%%%%%%%%%%%%%%%%%%%%%%%%%%%%%%%%%%%%%%

\section{Einleitung}

Programm-Autotuning findet zunehmend Anwendung in Domänen wie Hochleistungsrechnen oder
digitaler Bild-~und~Signalverarbeitung, zur Optimierung der entsprechenden Anwendung.
Durch die Verwendung eines Autotuners besteht die Möglichkeit, die Suche nach der
bestmöglichen Programm-Implementierung zu automatisieren. Anstatt ein Programm direkt
zu optimieren, beschreibt der Nutzer eine Menge möglicher Implementierungen, die 
systematisch von geeigneten Techniken durchsucht wird. Das Optimierungsverfahren
mittels Autotuner ist zumeist effizienter als die Optimierung von Hand, da deutlich größere
Suchräume verarbeitet werden können. \newline

Bei der Erstellung eines Autotuners steht oftmals die Verbesserung der Laufzeit
für ein spezifisches Programm im Vordergrund. OpenTuner bietet neben \texttt{time}\footnote{Zeit, Ausführungszeit}  
alternativ weitere Optimierungsziele, außerdem ist es möglich, mehrere Ziele bei der Optimierung 
zu verfolgen. \newline


Bestehende Autotuning-Frameworks wurden \iAllg zielgerichtet für den Einsatz in einer 
bestimmten Domäne entwickelt. \textsc{ATLAS} \cite{atlas} ist ein Projekt aus dem Bereich der Linearen Algebra,
\textsc{FFTW} \cite{fftw} verwendet Autotuning zum Lösen schneller Fourier-Transformationen.
Im Gegensatz dazu verfolgt OpenTuner den Ansatz, ein generelles System zur Erstellung 
von Autotunern für verschiedene Domänen einzuführen. Diese Flexibilität wird grundsätzlich durch eine stark
ausgeprägte Erweiterbarkeit und einen großen Funktionsumfang des Frameworks ermöglicht.
Das durch OpenTuner realisierte Optimierungsverfahren kann auf andere Rechnersysteme übertragen
und mit den dort existierenden Bedingungen wiederholt werden. \newline

Bei der Entwicklung eines Autotuning-Frameworks existieren drei grundsätzliche Herausforderungen.
Zunächst muss es möglich sein, für einen Problemfall eine \emph{passende Konfigurations-Repräsentation} wählen
zu können. Das umfasst die Darstellung der notwendigen Datenstrukturen und Bedingungen. Eine gute 
Repräsentation ist entscheidend für die Effizienz des Autotuners. \newline 

Ebenso kritisch kann die \emph{Größe des gültigen Suchraumes} sein, denn durch Kombination 
verschiedener Parameter sind schnell rießige Suchräume möglich. Eine vollständige Suche würde 
dann oftmals nicht abschließen,
daher bedarf es intelligenter Suchtechniken, welche nur einen kleinen Teil der Konfigurationsmenge
durchsuchen, um ein gutes Ergebnis zu erzielen. Die \emph{Beschaffenheit des Suchraumes} stellt
ebenfalls eine Herausforderung dar, da in vielen praktischen Anwendungen die Suchräume meist ein
hohes Maß an Komplexität besitzen.

\newpage

%%% ein wenig kürzen %%%

%%% section %%%%%%%%%%%%%%%%%%%%%%%%%%%%%%%%%%%%%%%%%%%%%%%%%%%%%%%%%%%%%%%%%%%%%%

\section{OpenTuner Framework}
Die Architektur des Frameworks ist in folgender Abbildung skizziert und lässt sich in die drei 
Komponenten \emph{Suche, Messung} und \emph{Ergebnis-Datenbank} gliedern.

\begin{figure}[h]
\begin{center}
\includegraphics[width=0.6\textwidth]{bilder/smdb}
\cite[S.~3]{OT-paper} \caption{Komponenten des OpenTuner Framework} 
\end{center}   
\end{figure}
\vspace{-.25cm} %%% cheat

Die Menge der Suchtechniken verwendet den Konfigurations-Manipulator, um Konfigurationen lesen und 
schreiben zu können. Ausgewählte Konfigurationen werden durch eine nutzerdefinierte Messfunktion
ausgeführt und anschließend ausgewertet. Die Datenbank dient dem Festhalten der Ergebnisse des Tuning-Vorgangs
und dem Informationsaustausch zwischen Suche und Messung.


\subsection{Suchtechniken}
OpenTuner stellt grundlegende Suchtechniken bereit, die auf viele Arten von 
Suchräumen anwendbar sind. Die Verwendung der Suchtechniken findet hierarchisch statt --
die Suchsteuerung spricht eine \texttt{root technique}\footnote{zentrale Technik (sog.~Meta-Technik),
ist den normalen Suchtechniken/Heuristiken übergeordnet} an, die dann während des Tuning-Vorgangs 
Tests auf eine Menge von \texttt{sub-techniques}\footnote{Standard-Suchtechniken:
\zB \texttt{Torczon hillclimbers, evolutionary mutation, pattern search}} verteilt.  \newline

Die \emph{Ensembles} sind ein Konzept von OpenTuner, welches die Kombination mehrerer Suchtechniken
ermöglicht. Einzelne Suchtechniken tauschen Ergebnisse über die Datenbank aus -- findet eine 
Suchtechnik eine gute Konfiguration, so können andere Techniken davon profitieren.
Suchtechniken bekommen ihre Testanteile dynamisch zugewiesen, abhängig davon wie erfolgreich eine
jeweils ist.

\subsection{Konfigurations-Manipulator}
Der Konfigurations-Manipulator stellt die Schnittstelle zwischen Suchtechniken und Konfigurationen
dar. Dieser verwaltet die Menge der Parameter inklusive der jeweiligen Belegung und regelt den Zugriff
(Lesen/Schreiben) durch die Suchtechniken.

\subsubsection{Parameter-Arten}

\begin{figure}[h]
\begin{center}
\includegraphics[width=0.85\textwidth]{bilder/paramtypes}
\cite[S.~5]{OT-paper} \caption{Hierarchie der integrierten Parameter-Arten} 
\end{center}
\end{figure}

Primitive Parameter repräsentieren numerische Werte einschließlich einem dafür
zulässigen und definierbaren Wertebereich\footnote{untere und obere Grenze des Parameters}. 
Zusätzlich existieren Varianten dieser Parameter,
deren konkreter Wert skaliert nach außen gegeben wird. Für ein Anwendungsszenario mit 
stark wachsenden Parametergrößen bietet sich die logarithmische Skalierung an. \newline

Der komplexe Parameter \texttt{Permutation} erstellt für eine gegebene Liste von Werten dessen 
Reihenfolge und besitzt die Funktionalität, zufällige Änderungen an dieser Ordnung vorzunehmen.
Für beide Parameter-Arten lassen sich bestehende Parameter erweitern oder neue erstellen.

\subsection{Optimierungsziele}

OpenTuner unterstützt mehrere Optimierungsziele, standardmäßig wird nach der Zeit optimiert.
Ergänzend zu \texttt{time} kann \ua \texttt{accuracy}\footnote{Genauigkeit, \zB Genauigkeit einer Berechnung}, 
\texttt{energy}\footnote{Energie, Energiebedarf eines bestimmten Systems} oder ein eigens vom 
Nutzer definiertes Ziel gewählt werden.
Darüber hinaus kann man bei der Optimierung mehr als ein Ziel gleichzeitig verfolgen und
diese nutzerspezifisch kombinieren (\zB \texttt{threshold accuracy  while
minimizing time}). 

\subsection{Suchen und Messen}

Das OpenTuner Framework besitzt die beiden Module \emph{Suchen} und \emph{Messen},
deren Kommunikation ausschließlich über die Ergebnis-Datenbank stattfindet.
Beim Entwurf von OpenTuner wurde sich bewusst dazu entschieden, diese zwei Komponenten
klar zu trennen. Diese Trennung hat \ua den Vorteil, dass mehrere Messungen parallel 
ausgeführt werden können -- da für den Großteil der Autotuning-Anwendungen der Messaufwand
den Suchaufwand deutlich übersteigt, ist die Unterteilung besonders sinnvoll.
Außerdem können dadurch Bestandteile wie das Messmodul einfach erweitert oder 
ausgetauscht werden, ohne das zugrundeliegende Framework ändern zu müssen.


%%% section %%%%%%%%%%%%%%%%%%%%%%%%%%%%%%%%%%%%%%%%%%%%%%%%%%%%%%%%%%%%%%%%%%%%%%

\section{Verwendung}
\label{Verwendung}
% Um mit OpenTuner einen Autotuner zu implementieren, sind folgende Schritte notwendig:
Folgende Schritte realisieren in OpenTuner einen Autotuner: 
\begin{enumerate}
  \item Suchraum definieren $\rightarrow$ \emph{configuration manipulator}
  \item Messfunktion erstellen (Konfigurationsauswertung) $\rightarrow$ \emph{run function}
  \item Festlegen des Optimierungsziels
\end{enumerate}
Weitere Optionen lassen sich von außen an den Autotuner übergeben. So kann 
mit \texttt{./tuner.py -{}-parallelism=2} spezifiziert werden, wieviele Konfigurationen
(hier~2) OpenTuner parallel auswerten soll. \newline

Da mit OpenTuner ein praxisorientierter Ansatz angestrebt wird, folgt nun eine Reihe von
Beispielen, in denen das Framework eingesetzt wird, um Anwendungen in unterschiedlicher 
Hinsicht zu optimieren. Für das jeweilige Beispiel wird der Problembereich beschrieben,
auf die Umsetzung des Autotuners eingegangen und die erzielten Resultate mit denen bisheriger
Autotuner verglichen.

\subsection{GCC/G++ Flags}
Für diesen Anwendungsfall steht die Auswahl und Kombination von Compileroptionen
im Vordergrund. Der Autotuner soll eine Menge von Flags und Parametern durchsuchen,
um die resultierende Ausführungszeit des kompilierten Zielprogramms zu minimieren.
Die unterstützten Flags erhält man mit \texttt{g++ --help=optimizers}, die 
Parameter inklusive der zulässigen Wertebereich aus \texttt{params.def} (gcc source code). \newline

Die Umsetzung des Autotuners lässt sich ideal mit OpenTuner realisieren, denn der
Suchraum kann durch Verwendung existierender Parameter ausreichend modelliert werden:
\begin{itemize}
  \item Optimierungslevel \texttt{(O0,$\ldots$,O3)} $\leftrightarrow$ \texttt{IntegerParameter~(min:~0, max:~3)}
  \item Flag $\leftrightarrow$ \texttt{EnumParameter~(on,~off,~default)}
  \item Parameter $\leftrightarrow$ \texttt{IntegerParameter~(min:~x, max:~y)}
\end{itemize}
Um die eigentliche Übersetzung mit einem Compiler durchführen zu können, sind konkrete
Zielprogramme notwendig. Für die Ergebnisse in diesem Beispiel wurde sich auf diese
Auswahl beschränkt: Schnelle Fourier-Transformation in C (\texttt{fft.c}) und 
ein Template zur Matrix-Multiplikation in \CC~(\texttt{matrixmultiply.cpp}). \newline

% kurze Überleitung zur Implementierung der Schritte 1-3 in Python
Die Implementierung des Autotuners geschieht anhand eines Python-Programms, da so die direkte Schnittstelle
zu OpenTuner spezifiziert ist. Es folgen nun repräsentative Code-Abschnitte \cite[S.~3]{OT-paper}, um eine
mögliche Umsetzung nachvollziehen zu können.  \newline

\lstinputlisting[language=Python]{etc/2830.py}
Zunächst erstellt man eine neue Variable zur Repräsentation des Suchraums: \texttt{manipulator}.
Das standardmäßig unterstützte Optimierungslevel des GCC wird dem Konfigurations-Manipulator
als \texttt{IntegerParameter}, inklusive der zulässigen Grenzen, hinzugefügt. \newline

\lstinputlisting[language=Python]{etc/gccflags.py}
Der Suchraum wird nun durch die einzelnen GCC-Flags als Menge von \texttt{EnumParameter} 
erweitert. \newline

\lstinputlisting[language=Python]{etc/gccparams.py}
Nach Einfügen der GCC-Parameter ist der Suchraum vollständig und es kann der nächste Schritt,
die Definition der Messfunktion, folgen. \newline

\lstinputlisting[language=Python]{etc/run.py}
Um eine Messung ausführen zu können, ist dafür zunächst eine Konfiguration notwendig.
Diese Konfiguration wird durch die String-Variable \texttt{gcc\_cmd} repräsentiert, welche
aus \texttt{cfg} schrittweise konkrete Parameterbelegungen hinzugefügt bekommt. \newline

\lstinputlisting[language=Python]{etc/flagsonoff.py}
Nun wird der Konfiguration die Information übertragen, ob das jeweilige GCC-Flag
aktiviert werden soll oder nicht, \zB~'-falign-functions' \vs~'-fno-align-functions'.

\lstinputlisting[language=Python]{etc/params.py}
Für jeden Parameter aus \texttt{GCC\_PARAMS} wird die Bezeichnung und der
konkrete Wert an \texttt{gcc\_cmd} übergeben, \zB~'-{}-param early-inlining-insns=512'.
Das Erstellen der zu testenden Konfiguration ist damit abgeschlossen. \newline

\lstinputlisting[language=Python]{etc/measure.py}
Um nun herausfinden zu können, wie gut die erstellte Konfiguration tatsächlich ist, 
wird diese (\texttt{gcc\_cmd}) kompiliert und anschließend ausgeführt: \texttt{'./tmp.bin'}.
Die ursprüngliche Intention des Tuning-Vorgangs war es, die Ausführungszeit des 
Executable zu minimieren. Deshalb wird das entstandene Messergebnis aus der Datenbank,
in Spalte/Attribut \texttt{time} (Festlegen des Optimierungsziel), abgefragt und zurückgegeben. \newline

Dieser Autotuner wurde in vollständiger Form\footnote{zu finden im Framework: 
examples/gccflags/gccflags.py} auf die Beispiele \texttt{fft.c} 
und \texttt{matrixmultiply.cpp} angewandt, es wurden immer 30 Tuning-Runs\footnote{wiederholte Anwendung 
des Autotuners} vollzogen\footnote{Testsystem: \texttt{Xeon E5320, gcc 4.7.3-1ubuntu1}}.

\begin{figure}[h]
\begin{center}
\includegraphics[width=0.8\textwidth]{bilder/fft_cropped}
\cite{OT-paper} \caption{\texttt{fft.c}}
\end{center}
\end{figure}

Bereits im zweiten Durchlauf des Autotuners wird die bisherige Bestzeit (\texttt{gcc -O2}) 
erreicht und kontinuierlich verbessert. Der Code des Zielprogramms \texttt{fft.c} war bereits
hochoptimiert\footnote{cache aware tiling, threading \cite[S.~6]{OT-paper}} 
-- dennoch war durch OpenTuner schließlich eine zeitliche Verbesserung von $1.15 \times$  zu verzeichnen.

\begin{figure}[h]
\begin{center}
\includegraphics[width=0.8\textwidth]{bilder/mm_cropped}
\cite{OT-paper} \caption{\texttt{matrixmultiply.cpp}}
\end{center}
\end{figure}

Bei dem Beispiel der Matrix-Multiplikation wird die Bestzeit ebenfalls schnell unterboten.
Hier wird die Verbesserung aber besonders deutlich: der resultierende Speedup beträgt
$2.82 \times$. Dieses Ergebnis ist überdurchschnittlich, was letztlich auch darauf beruht,
dass die ursprüngliche Implementierung (\texttt{matrixmultiply.cpp}) kaum optimiert war.

\subsection{High Performance Linpack Benchmark (HPL)}
Der High Performance Linpack Benchmark \cite{hpl} dient zur Bestimmung der 
\texttt{floating point performance} unterschiedlicher Computersysteme und gilt als 
Kriterium für die Wahl der Top 500 Supercomputer \cite{top500}.
Zur Bestimmung der Performance eines Systems misst der Benchmark die Geschwindigkeit 
für das Lösen großer Linearer Gleichungssysteme. \newline

Es müssen etwa 15 verschiedene Parameter
optimiert werden, dabei hat die Matrix-Blockgröße den stärksten Einfluss.
HPL besitzt selbst einen eingebauten Autotuner, der als Suchtechnik 
\emph{vollständige Suche} einsetzt. Die folgende Grafik zeigt die erreichte Leistung\footnote{
Testsystem: \texttt{Intel Sandy Bridge quad-core, GCC 4.5, Intel  Math  Kernel  Library  (MKL)  11.0}}
mit OpenTuner im Vergleich zur optimierten HPL-Implementierung von Intel\footnote{\url{http://software.intel.com/en-us/articles/intel-math-kernel-library-linpack-download}}. \newline


\begin{figure}[h]
\begin{center}
\includegraphics[width=0.8\textwidth]{bilder/hpl}
\cite{OT-paper} \caption{HPL}
\end{center}
\end{figure}

Diesmal sind deutlich mehr Tuning-Runs notwendig, um die Bestzeit zu erreichen \bzw zu unterbieten.
Nach mehr als $1200$ Sekunden jedoch, erreicht OpenTuner eine neue günstigste Execution Time,
welche gegen Ende nochmals leicht verbessert wird. Hier gilt zu beachten, dass der resultierende
Speedup minimal ausfällt, aber OpenTuner im Vergleich zum bisherigen Autotuner nur einen 
sehr kleinen Anteil des Suchraums benötigt, um ein vergleichbar gutes \bzw besseres Ergebnis 
zu erzielen.

\subsection{Super Mario}
Um zu verdeutlichen, wie flexibel und vielseitig einsetzbar das OpenTuner Framework tatsächlich ist,
folgt nun ein sehr demonstratives Anwendungsbeispiel. Denn OpenTuner kann verwendet werden,
um das erste Level des Videospiels Super Mario Bros. erfolgreich zu absolvieren.
Das Spielziel wird durch eine Abfolge von Tasten-Eingaben erreicht, welche die eigentlichen
Konfigurationen im Autotuner darstellen. \newline

Die Zielfunktion besteht nun darin, die Pixel-Anzahl, welche Mario in Richtung Zielflagge (rechts)
zurückgelegt hat, zu maximieren. Ein NES-Emulator\footnote{\url{www.fceux.com/web/home.html}}
nimmt die konkreten Eingaben an, die Wiedergabe erfolgt schneller als in Echtzeit. Zudem können
mehrere Instanzen des Emulators parallel ausgewertet werden, wodurch die Ergebnissuche 
deutlich beschleunigt wird. \newline

Das vollständige Setup (Autotuner, NES-Emulator und -Rom) habe ich auf meinem 
Heimsystem\footnote{Core i5-3570K, Ubuntu 16.10} getestet und kann daher 
bestätigen, dass es möglich ist, OpenTuner sogar für ein Videospiel erfolgreich
einsetzen zu können. Auf besagtem System lief der Autotuner mit 8 Instanzen des
Emulators flüssig (\texttt{./mario.py -{}-parallelism=8 \ldots}) und nach etwa drei Minuten
war bereits der erste erfolgreiche Durchlauf\footnote{Mein Replay: \url{www.youtube.com/watch?v=5c7Ytwyywtk}} zu verzeichnen. \newline

Im weiteren Verlauf des Tuning-Vorgangs entstehen immer bessere Durchläufe, jedoch
hat dieses automatisierte System einen entscheidenden Nachteil: im Gegensatz zum menschlichen
Spieler, sind keinerlei Informationen über die wirkliche Spielumgebung vorhanden.
Denn im ersten Level gibt es eine Abkürzung, welche durch einen Tastendruck nach 
unten erreicht wird -- der Autotuner ist dafür nicht konfiguriert und wird diesen Weg
nicht wählen.


\subsection{Ähnliche Anwendungen}
Das Projekt Halide \cite{halide} stellt eine Sprache und Compiler für den Bereich der 
Bildverarbeitung, insbesondere dem Einsatz gestaffelter Rendering-Pipelines, dar.
Die Ausführungsreihenfolge in diesen Pipelines kann mit OpenTuner modelliert und 
optimiert werden, \texttt{PermutationParameter} und \texttt{ScheduleParameter}
können hier ideal verwendet werden. Die Pipeline-Schedule bestimmt die Performance
von Halide maßgeblich, daher ist eine Verbesserung der Ausführungszeit wünschenswert. \newline

PetaBricks \cite{petab} nutzt das Konzept der \emph{algorithmic selectors}, also die 
Synthese verschiedener Algorithmen (\emph{poly-algorithms}) zur Lösung eines bestimmten algorithmischen Problems.
OpenTuner unterstützt die Verwendung von Selektoren (\texttt{SelectorParameter})
und kann damit für eine konkrete Bedingung einen bestimmten Algorithmus auswählen. 
Eine Anwendung könnte darin bestehen, geeignete Suchalgorithmen für konkrete
Eingabegrößen, zu komibinieren.


%%% section %%%%%%%%%%%%%%%%%%%%%%%%%%%%%%%%%%%%%%%%%%%%%%%%%%%%%%%%%%%%%%%%%%%%%%

\section{Zusammenfassung}
OpenTuner wurde hauptsächlich zwischen 2014 und 2015 entwickelt und steht auf GitHub 
zur freien Verfügung \cite{ghub}. Das Projekt wurde in vergangener Zeit nicht aktiv
weiterentwickelt, was auch durch die Aussage der Autoren zu begründen ist, dass das
Framework \emph{out-of-the-box} funktioniert. Durch die bewusst flexible Architektur
ist es den Nutzern möglich, beliebig neue Funktionalitäten hinzuzufügen und
existierende Bestandteile zu modifizieren. \newline

Ziel des Projekts ist es, die Erstellung neuer Autotuner zu vereinfachen, bessere
Optimierungsergebnisse zu erzielen und das Tuning-System auf lange Sicht wartbar
zu gestalten. OpenTuner steht in Konkurrenz zu bisherigen Verfahren in 
bestehenden Domänen\footnote{\bspw Lineare Algebra, Hochleistungsrechnen, Signalverarbeitung}
und ist außerdem verwendbar in neuen Bereichen, was auch durch die Anwendungsbeispiele in 
Abschnitt \ref{Verwendung} deutlich wird.

\subsection{Fazit}
Das OpenTuner Framework zeichnet sich in erster Linie durch seine hohe Flexibilität
und Erweiterbarkeit aus. Bisherige Verfahren funktionieren meist nur für einen speziellen
Anwendungsfall, OpenTuner hat nicht nur den Vorteil auf verschiedenste Szenarien anwendbar zu sein,
sondern ist auch stark konfigurierbar, um \bspw neue Suchtechniken zu unterstützen. \newline

Der Nutzer kann anstatt ein Programm direkt und mit großem Aufwand zu optimieren,
eine Art \emph{Modellierung} des Problembereichs vornehmen, was in einer deutlich
einfacheren und klaren Handhabung resultiert. Dennoch ist natürlich grundlegendes
Wissen über Autotuning \bzw die Domäne notwendig. \newline

Das Projekt stellt über das Framework und der zugehörigen Website \cite{ot} eine
Vielzahl an Beispielanwendungen bereit, wodurch sich der praktische Einsatz sehr
gut nachvollziehen lässt. Um einen ersten Autotuner zu erstellen, existiert eine
ausreichend gute Dokumentation, je fortgeschrittener die Funktionalität ist,
desto schwieriger wird es aber, dafür umfassende Beschreibungen zu finden. \newline

Eine weitere Besonderheit besteht darin, dass alternativ zu \texttt{time} 
standardmäßig weitere Optimierungsziele unterstützt werden, neue
Ziele definierbar sind und mehrere in Kombination verwendet werden können.
Entscheidendes Kriterium ist letztlich natürlich, dass OpenTuner 
effektiv \emph{bessere Ergebnisse} als bisherige Autotuner \emph{erreichen kann}. \newline

Zu bedenken gilt dennoch, dass der mögliche Optimierungsgrad stark variiert, das ist 
vom konkreten Anwendungsfall und vor allem von der Qualität des Autotuners abhängig.
Es kann mitunter vorkommen, dass die benötigte Zeit für den Optimierungsvorgang
so groß ist, dass eine Verwendung bedenklich ist.
Daher gilt also abzuwägen, ob der Zeitaufwand für den Entwurf und Einsatz des Autotuners
und der effektive Nutzen der Optimierung, in einem vertretbaren Verhältnis stehen.
\newpage


%%% section %%%%%%%%%%%%%%%%%%%%%%%%%%%%%%%%%%%%%%%%%%%%%%%%%%%%%%%%%%%%%%%%%%%%%%

\section{Literaturverzeichnis}
% abc

\begin{thebibliography}{9}

  
\bibitem{OT-paper} Jason Ansel, Shoaib Kamil, Kalyan Veeramachaneni, Jonathan Ragan-Kelley, Jeffrey Bosboom, Una-May O'Reilly, Saman Amarasinghe. \emph{OpenTuner: An Extensible Framework for Program Autotuning.} 2014.

\bibitem{} Shoaib Kamil. \emph{Hands on session.} Programming Language Design and
Implementation 2015

\bibitem{ghub} \url{www.github.com/jansel/opentuner}, Januar 2017

\bibitem{ot} \url{http://www.opentuner.org}, Januar 2017
   
\bibitem{atlas} \url{http://math-atlas.sourceforge.net/}, Januar 2017

\bibitem{fftw} \url{http://www.fftw.org/}, Januar 2017
    

\bibitem{hpl} \url{http://www.netlib.org/benchmark/hpl/}, Januar 2017

\bibitem{top500}  Top500, \url{www.top500.org}, Januar 2017

\bibitem{halide} \url{http://halide-lang.org/}, Januar 2017

\bibitem{petab}  \url{http://projects.csail.mit.edu/petabricks/}, Januar 2017

\end{thebibliography}
\end{document}
