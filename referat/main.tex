%%%%%%%%%%%%%%%%%%%%%%%%%%%%%%%%%%%%%%%%%%%%%%%%%%%%%%%%%%%%%%%%%%%%%%%%%%%%%%%
%                                                                             %
% Anwendungsbeispiel für das Beamer-Template TU Chemnitz                      %
% (c) Mario Haustein (mario.haustein@hrz.tu-chemnitz.de), 2013-2014           %
%                                                                             %
%%%%%%%%%%%%%%%%%%%%%%%%%%%%%%%%%%%%%%%%%%%%%%%%%%%%%%%%%%%%%%%%%%%%%%%%%%%%%%%

\usepackage[utf8]{inputenc}
\usepackage{babel}



% TUC-Templates laden.
\usetheme[fakcolor=if]{tuc2014}
\mode<article>{\usepackage{beamerarticletuc2014}}


%
% Weitere Anpassungen nach Bedarf.
%


% Metadaten
\title{Professur Praktische Informatik}
\subtitle{OpenTuner: An Extensible Framework for Program Autotuning}
%\subtitle{Hauptseminar}
\author{Matthias Tietz}

%\normalsize{Betreuer: Dr. Michael Hofmann}



\date{\today}
\institute[TUC]

\begin{document}
    \tucthreeheadlines

    \begin{frame}
        \titlepage
    \end{frame}

    \tucnarrowframe
    \begingroup
% kein logo
%    \logo{\includegraphics[width=\hsize]{Grafiken/logo}}

    \begin{frame}
        \frametitle{Gliederung}
        \tableofcontents
    \end{frame}

    \endgroup
    \tucwideframe
    \tuctwoheadlines

    %
    % Inhalt
    %

    \section{Einleitung - Warum OpenTuner?}

%     \subsection{Soziales}

    \begin{frame}
    \begin{itemize}
      \item Viele Gründe sprechen dafür
    \end{itemize}
    \end{frame}

\end{document}
