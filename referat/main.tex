%%%%%%%%%%%%%%%%%%%%%%%%%%%%%%%%%%%%%%%%%%%%%%%%%%%%%%%%%%%%%%%%%%%%%%%%%%%%%%%
%                                                                             %
% Anwendungsbeispiel für das Beamer-Template TU Chemnitz                      %
% (c) Mario Haustein (mario.haustein@hrz.tu-chemnitz.de), 2013-2014           %
%                                                                             %
%%%%%%%%%%%%%%%%%%%%%%%%%%%%%%%%%%%%%%%%%%%%%%%%%%%%%%%%%%%%%%%%%%%%%%%%%%%%%%%

\usepackage[utf8]{inputenc}
\usepackage{babel}
\usepackage{floatflt}
\usepackage{float}
\usepackage{graphics}
\usepackage{graphicx}
\usepackage{listings}
\usepackage{color}
\usepackage{url}
\usepackage{hyperref}
\usepackage{wrapfig}

% TUC-Templates laden.
\usetheme[fakcolor=if]{tuc2014}
\mode<article>{\usepackage{beamerarticletuc2014}}

%
% Weitere Anpassungen nach Bedarf.
%

% Metadaten

% \titlegraphic{\includegraphics[width=4cm]{bilder/opentunerlogo.png}}
\title{OpenTuner: An Extensible Framework \\for Program Autotuning}
\author{Matthias Tietz}
\date{10.01.2017}
\institute[TUC]{\small  Hauptseminar\\Bachelor Informatik}


\tucurl{www.tu-chemnitz.de}

\begin{document}

    \tucthreeheadlines

    \begingroup
      \setbeamertemplate{tuc3 headline 2}{Hauptseminar - Bachelor Informatik}
      \setbeamertemplate{tuc3 headline 3}{Matthias Tietz}
      \begin{frame}
        \titlepage
    \end{frame}
    \endgroup

    \begingroup
      \setbeamertemplate{tuc3 headline 2}{Hauptseminar - Bachelor Informatik}
      \setbeamertemplate{tuc3 headline 3}{Gliederung}
    \begin{frame}
        \begin{wrapfigure}{r}{0.65\textwidth}
          \begin{center}
            \includegraphics[width=0.5\textwidth]{bilder/opentunerlogo2}
          \end{center}
        \end{wrapfigure}
%         \large{Gliederung}
        \normalsize \tableofcontents  
    \end{frame}
    \endgroup

    \tuctwoheadlines

    %
    % Inhalt
    %
    
%%%%% section %%%%%%%%%%%%%%%%%%%%%%%%%%%%%%%%%%%%%%%%%%%%%%%%%%%%%%%%%

    \section{Einleitung}
    
  \begin{frame}
    \tableofcontents[currentsection]
  \end{frame}

%%%%% new frame %%%%%%%%%%%%%%%%%%%%%%%%%%%%%%%%%%%%%%%%%%%%%%%%%%%%%%%   

  \begingroup
  \setbeamertemplate{tuc2 headline 2}{Motivation}
    \begin{frame}
    
    \textbf{Paper}
    \begin{itemize}
    
        \item OpenTuner: An Extensible Framework for Program Autotuning
        
        \item Jason Ansel, Shoaib Kamil, Kalyan Veeramachaneni, Jonathan Ragan-Kelley, Jeffrey Bosboom, Una-May O'Reilly, Saman Amarasinghe (MIT 2014)
        % PetaBricks
        
%         \item Massachusetts Institute of Technology, 2014 
    \end{itemize}
    
    \text{}
    
    \textbf{Motivation}
        
    \begin{itemize}
%       \item Optimierung rechenintensiver Programme
      % findet Anwendung in Domänen wie HPC und Grafik-/Signalverarbeitung (PS, 3D, image processing)
      % (hoher Rechenaufwand) -> Zeitgewinn ... hand-optimierte AT ..
      
      \item Tuning-Prozess: automatisierbar, effizient?
      % Suche nach optimaler Programm-Implementierung
      % große Suchräume, bisherige AT beschneiden oft SR !
      % Suchraum = Parameter, mögliche Implementierungen
      % per Hand schwer machbar bzw. sehr ineffizient
      
      \item bestehende Autotuning-Frameworks domänenspezifisch (ATLAS, FFTW)
      % lineare Algebra, FFT
      % referenz / link auf ATLAS, FFTW
      
      \end{itemize}
      
      \text{OpenTuner:}
    \begin{itemize}
      \item komfortables Optimierungsverfahren
      
      \item mehrere Optimierungsziele kombinierbar, hohe Erweiterbarkeit
      % Tradeoff Performance - Genauigkeit
      
       \item Performance-Portabilität
      % AT-Prozess einfach auf neuer Maschine ausführbar
                  
    \end{itemize}
    \end{frame}
    \endgroup
    
%%%%% new frame %%%%%%%%%%%%%%%%%%%%%%%%%%%%%%%%%%%%%%%%%%%%%%%%%%%%%%%   
  \begingroup
  \setbeamertemplate{tuc2 headline 2}{Wichtige Begriffe}
  \begin{frame}
    \textbf{Wichtige Begriffe:}
    
  \begin{itemize}

    \item \textbf{Parameter}: Variable eines bestimmten Typs 
    % Modellierung
    % Programm XY -> welche Parameter sollen modelliert werden? spätere Folie dazu mehr
    
    \item \textbf{Konfiguration}: konkrete Wert-Zuweisung (Parameter)
    \item \textbf{Konfigurations-Manipulator}: Bearbeiten der Parameter \newline
    % Funktion
    
    \item \textbf{Suchraum}: zu durchsuchende Menge von Konfigurationen
    % mögliche Implementierungen
    
    \item \textbf{Suchtechnik}: Suchraum erforschen, Anfragen für Messung
    % folie mit grafik schön zu sehen
%     \item \textbf{Suchprozess}: Lesen/Schreiben von Konfigurationen \newline
    
%     \item \textbf{Messprozess}: konkrete Konfiguration ausführen
    \item \textbf{Messresultat}: Ergebnis der Messung, abhängig von Optimierungsziel(en)
    \item \textbf{Ergebnisdatenbank}: Festhalten aller Ergebnisse des Tuning-Vorgangs
    
  \end{itemize}
  \end{frame}
  \endgroup
        
%%%%% new frame %%%%%%%%%%%%%%%%%%%%%%%%%%%%%%%%%%%%%%%%%%%%%%%%%%%%%%%   
    \begingroup
    \setbeamertemplate{tuc2 headline 2}{Herausforderungen}
    \begin{frame}
    \textbf{Herausforderungen bei der Entwicklung eines Autotuning-Frameworks:}
    
    \begin{itemize}
       
    \item 1. Eine passende Konfigurations-Repräsentation
        \begin{itemize}
          \item Darstellung der Datenstrukturen und Bedingungen
          \item entscheidend für Effizienz des Autotuners \newline
          % Quali der Repr.
          
        \end{itemize}
        
    \item 2. Größe des gültigen Suchraumes
        \begin{itemize}
          \item rießige Konfigurationsräume möglich $\rightarrow$  intelligente Suchtechniken
          
          \item Kürzen des Suchraumes $\rightarrow$ Verlieren guter Lösungen \newline
          % (bei bisherigen Autotunern gängige Praxis, da vollständige Suche)
          
        \end{itemize}
        
    \item 3. Beschaffenheit des Suchraumes
      \begin{itemize}
        \item Suchräume in der Praxis meist sehr komplex 
        \item domänenspezifische Suchtechniken notwendig
        % um optimale Lösung effizient zu ermitteln
        
      \end{itemize}
    \end{itemize}
        
    \end{frame}
    \endgroup
    
%%%%% new frame %%%%%%%%%%%%%%%%%%%%%%%%%%%%%%%%%%%%%%%%%%%%%%%%%%%%%%%   

%     \begin{frame}
%     
%     \textbf{Eigenschaften von OpenTuner:}   
%     \begin{itemize}
%       \item Erstellen domänenspezifischer und multi-objective Programm-Autotuner
%       \item vollständig anpassbare Konfigurations-Repräsentation
%       \item erweiterbare Repräsentation für Suchtechniken und Datentypen
%       \item Kombination mehrerer Suchtechniken (\textit{Ensembles}), dynamische Zuweisung der
%       Testanteile für die jeweiligen Suchtechniken
%       \item einfache Schnittstelle zur Kommunikation mit dem zu optimierenden Programm           
%     \end{itemize}
%     \end{frame}
    
%%%%% new frame %%%%%%%%%%%%%%%%%%%%%%%%%%%%%%%%%%%%%%%%%%%%%%%%%%%%%%%   
%%%%% section %%%%%%%%%%%%%%%%%%%%%%%%%%%%%%%%%%%%%%%%%%%%%%%%%%%%%%%%%
  \section{OpenTuner Framework}
  
  \begin{frame}
    \tableofcontents[currentsection]
  \end{frame}

%%%%% new frame %%%%%%%%%%%%%%%%%%%%%%%%%%%%%%%%%%%%%%%%%%%%%%%%%%%%%%%  
  
  \begingroup
  \setbeamertemplate{tuc2 headline 2}{Organisation}
  \begin{frame}
  
  
  \begin{figure}[ht]
    \centering	      
    \includegraphics[width=0.8\textwidth]{bilder/smdb.pdf}
    \label{smdb}
  \end{figure}

  \end{frame}
  \endgroup
    
%%%%% new frame %%%%%%%%%%%%%%%%%%%%%%%%%%%%%%%%%%%%%%%%%%%%%%%%%%%%%%%
%  \subsection{Verwendung}
  \begingroup
  \setbeamertemplate{tuc2 headline 2}{Module}
  \begin{frame}
  \textbf{Verwendung}
  \begin{itemize}
    \item 1. Suchraum definieren 
    \item 2. \texttt{run()}-Methode definieren: Auswerten der Konfiguration $\rightarrow$ Ergebnis
    \item 3. Festlegen des Optimierungsziels
    \item Umsetzung: Python-Programm (OpenTuner API) \newline
    % Framework ist ausschließlich in Python geschrieben
    % Bsp. später darauf eingehen, sieht man es besser
    
  \end{itemize}
  \textbf{Suchtechniken}
  \begin{itemize}
    \item OpenTuner stellt grundlegende Suchtechniken bereit
    \item Ausführen mehrerer Suchtechniken (Ensembles)
    \item Austausch von Ergebnissen über Datenbank
%     $\rightarrow$ Profitieren von Verbesserungen anderer Suchtechniken

    \item dynamische Testzuweisung anhand Erfolg der Suchtechniken

    
  \end{itemize}
  \end{frame}
  \endgroup
    
%%%%% new frame %%%%%%%%%%%%%%%%%%%%%%%%%%%%%%%%%%%%%%%%%%%%%%%%%%%%%%%
  \begingroup
  \setbeamertemplate{tuc2 headline 2}{Module}
  \begin{frame}
    \textbf{Parameter-Arten}
    \begin{figure}[ht]
      \centering	      
      \includegraphics[scale=1.3]{bilder/paramtypes}
      \label{paramtypes}
    \end{figure}
    
  \textbf{Primitive Parameter}
  \begin{itemize}
    \item numerische Werte mit Unter-/Obergrenze
    \item \texttt{Float} und \texttt{LogFloat(-Int)} gleiche Repräsentation, Werte skaliert
%     \item Grund: Effekt der Wertänderung bei steigender Parametergröße
    % ähnlich bei PowerOfTwo
    
  \end{itemize}
  \end{frame}
  \endgroup
  
%%%%% new frame %%%%%%%%%%%%%%%%%%%%%%%%%%%%%%%%%%%%%%%%%%%%%%%%%%%%%%%
  \begingroup
  \setbeamertemplate{tuc2 headline 2}{Module}
  \begin{frame}
    \textbf{Komplexe Parameter}
    \begin{itemize}

%       \item \texttt{Boolean, Switch, Enum} als komplex. Parameter, Darstellung
%       als ungeordnete Sammlung $\rightarrow$ es existiert kein Gradient
      % (wie bei prim. Param.)
      
      \item \texttt{Permutation:} Liste von Werten inkl. Reihenfolge-Manipulator
      \item \texttt{Schedule, Selector} \newline
      % ist Sonderfall von \texttt{Permutation} : topolog. Sortierung nach jeder Änderung
      % Selector: Mapping von Integer-Input auf Enum-Type (Darstellung als Baum)
      
    \end{itemize}
  
  \textbf{Optimierungsziele}
    \begin{itemize}
      \item OpenTuner unterstützt mehrere Ziele, default: \texttt{time}
      \item \texttt{accuracy, energy, size, confidence} oder ein nutzerdef. Ziel
      \item Ziel-Kombination: \texttt{time-accuracy}
      % Genauigkeit einhalten, gleichzeitig Zeit minimieren
    \end{itemize}
  \end{frame}
  \endgroup
    
%%%%% new frame %%%%%%%%%%%%%%%%%%%%%%%%%%%%%%%%%%%%%%%%%%%%%%%%%%%%%%%
%%%%% Section %%%%%%%%%%%%%%%%%%%%%%%%%%%%%%%%%%%%%%%%%%%%%%%%%%%%%%%%%
    \section{Anwendungsbeispiele}
    
    \begin{frame}
      \tableofcontents[currentsection]
    \end{frame}
    
%%%%% new frame %%%%%%%%%%%%%%%%%%%%%%%%%%%%%%%%%%%%%%%%%%%%%%%%%%%%%%%
    
    \begingroup
    \setbeamertemplate{tuc2 headline 2}{GCC/G++ Flags}
    \begin{frame}
      
    \textbf{GCC/G++ Flags}
      \begin{itemize}
        \item klassischer Parameter-Autotuner
        \item unterstützte Flags: \texttt{g++ --help=optimizers}
        \item Parameter inkl. zulässiger Wertebereiche: \texttt{params.def} (gcc source code)
        
%         \item Autotuner-Implementierung: 
%         \begin{itemize}
%           \item \texttt{configuration manipulator}
%           \item \texttt{run()} 
%           \item Optimierungsziel        
%         \end{itemize}
        
        \item Zielprogramme zur Laufzeit-Optimierung:
        \begin{itemize}
          \item \texttt{fft.c} - Schnelle Fourier-Transformation
          \item \texttt{matrixmultiply.cpp} - Matrix-Multiplikation
        \end{itemize}        
      \end{itemize}
    \end{frame}
    \endgroup
        
%%%%% new frame %%%%%%%%%%%%%%%%%%%%%%%%%%%%%%%%%%%%%%%%%%%%%%%%%%%%%%%
    \begingroup
    \setbeamertemplate{tuc2 headline 2}{GCC/G++ Flags}
  \begin{frame}
    \begin{figure}[ht]
      \centering	      
      \includegraphics[width=0.8\textwidth]{bilder/ot-3-1a}
      \label{gccpy1a}
    \end{figure}
  \end{frame}
  \endgroup
  
%%%%% new frame %%%%%%%%%%%%%%%%%%%%%%%%%%%%%%%%%%%%%%%%%%%%%%%%%%%%%%%
  \begingroup
  \setbeamertemplate{tuc2 headline 2}{GCC/G++ Flags}
  \begin{frame}
    \begin{figure}[ht]
      \centering	      
      \includegraphics[width=0.75\textwidth]{bilder/ot-3-1}
      \label{gccpy1}
    \end{figure}
  \end{frame}
\endgroup
  
%%%%% new frame %%%%%%%%%%%%%%%%%%%%%%%%%%%%%%%%%%%%%%%%%%%%%%%%%%%%%%%
    \begingroup
  \setbeamertemplate{tuc2 headline 2}{GCC/G++ Flags}
  \begin{frame}
    \begin{figure}[ht]
      \centering	      
      \includegraphics[width=0.65\textwidth]{bilder/ot-3-2}
      \label{gccpy2}
    \end{figure}
  \end{frame}
  \endgroup

%%%%% new frame %%%%%%%%%%%%%%%%%%%%%%%%%%%%%%%%%%%%%%%%%%%%%%%%%%%%%%%
 \begingroup
  \setbeamertemplate{tuc2 headline 2}{GCC/G++ Flags}
  \begin{frame}
      \begin{figure}[ht]
      \centering	      
      \includegraphics[width=0.8\textwidth]{bilder/fft.pdf}
      \label{gccResults}
    \end{figure}
  \end{frame} 
  \endgroup
  
%%%%% new frame %%%%%%%%%%%%%%%%%%%%%%%%%%%%%%%%%%%%%%%%%%%%%%%%%%%%%%%
 \begingroup
  \setbeamertemplate{tuc2 headline 2}{GCC/G++ Flags}
  \begin{frame}
      \begin{figure}[ht]
      \centering	      
      \includegraphics[width=0.8\textwidth]{bilder/mm.pdf}
      \label{gccResults}
    \end{figure}
  \end{frame} 
  \endgroup
  
  
%%%%% new frame %%%%%%%%%%%%%%%%%%%%%%%%%%%%%%%%%%%%%%%%%%%%%%%%%%%%%%%
 \begingroup
  \setbeamertemplate{tuc2 headline 2}{HPL}
\begin{frame}
  \textbf{High Performance Linpack Benchmark (HPL)}
  \begin{itemize}
    \item Bestimmen der \texttt{floating point performance} von Computern
    %  kleine Mehrprozessorsysteme - rießige Clustersysteme
    
    \item Kriterium für Wahl der Top 500 Supercomputer
    \item HPL misst Geschwindigkeit für Lösen großer LGS
    % linearen Gleichungssystem
    
    \item ca. 15 verschiedene Parameter
    % restlichen Parameter großteils Enum, Switch (diskrete Entsch. in verwendeten Algorithmen)
    
    \item größter Performance-Einfluss: Matrix-Blockgröße
    % Verteilung der Matrizen auf Prozessoren anhand der Anzahl der CPUs
   
    \item HPL besitzt eingebauten Autotuner (vollst. Suche)
        
    \item OpenTuner im Vergleich zu HPL-Implementierung von Intel
    % "im Vergleich zu Intel's HPL-Implementierung"
    
  \end{itemize}  
\end{frame}
\endgroup

%%%%% new frame %%%%%%%%%%%%%%%%%%%%%%%%%%%%%%%%%%%%%%%%%%%%%%%%%%%%%%%
 \begingroup
  \setbeamertemplate{tuc2 headline 2}{HPL}
\begin{frame}
    \begin{figure}[ht]
      \centering	      
      \includegraphics[width=0.9\textwidth]{bilder/hpl.pdf}
      \label{hpl}
    \end{figure}
\end{frame}
\endgroup
    
%%%%% new frame %%%%%%%%%%%%%%%%%%%%%%%%%%%%%%%%%%%%%%%%%%%%%%%%%%%%%%%
%  \begin{frame}
%   \textbf{b) Unitary Matrices}
   
%     \begin{itemize}
%       \item Synthetisierung von Matrizen in optimaler Zeit
%       \item \textbf{bisher} traditioneller Autotuner zur Programmoptimierung
%       \item \textbf{hier} Suche als Subroutine zur Laufzeit
%       \item das Problem hat fixes Set von Operatoren (Controls), repräsentiert als Matrizen
%       \item Ziel: Finden einer Sequenz von Operatoren, sodass Matrixmultiplikation die Zielmatrix ergibt
%       \item Zielfunktion: Abstand (Genauigkeitswert) des Produktes der aktuellen Sequenz zum Ziel (trace fidelity)
%     \end{itemize}
%   \end{frame}
%   
%%%%% new frame %%%%%%%%%%%%%%%%%%%%%%%%%%%%%%%%%%%%%%%%%%%%%%%%%%%%%%%
% \begin{frame}
%     \begin{figure}[ht]
%       \centering	      
%       \includegraphics[width=0.9\textwidth]{bilder/unitary.pdf}
%       \label{gccpy1a}
%     \end{figure}
% \end{frame}
% 

    
%%%%% new frame %%%%%%%%%%%%%%%%%%%%%%%%%%%%%%%%%%%%%%%%%%%%%%%%%%%%%%%

%%%%%%%%%%%%%%%%%%%%%%%%%%%%%%%%%%%%
%%%%%%%%%%%%  TODO:  %%%%%%%%%%%%%%%
%%%%%%%%%%%%%%%%%%%%%%%%%%%%%%%%%%%%

%   \begin{frame}

%         \textbf{3. Konkrete Anwendungen der Entwickler} \newline

%    \textbf{Matrixmultiplikation}
%    
%     \begin{itemize}
%       \item reicht evtl. auch nur ein eigenens bsp?
%       \item abc abc abc
%     \end{itemize}
%   \end{frame}

%%%%% new frame %%%%%%%%%%%%%%%%%%%%%%%%%%%%%%%%%%%%%%%%%%%%%%%%%%%%%%%
 \begingroup
  \setbeamertemplate{tuc2 headline 2}{Ähnliche Anwendungen}
\begin{frame}
  \textbf{Ähnliche Anwendungen des Frameworks} \newline
  
 
  \textbf{Halide}
  \begin{itemize}
    \item Sprache und Compiler für Bildverarbeitung
    \item Ausführungsreihenfolge in Grafik-Pipeline
    \item Pipeline-Schedule bestimmt die Performance
   \end{itemize}
      
   \text{}
   
   \textbf{PetaBricks}
   \begin{itemize}
     \item Kombination versch. Algorithmen \texttt{(algorithmic selectors)}
     \item \texttt{block sizes, thread counts, iteration counts} \ldots
   \end{itemize}
\end{frame}
\endgroup
    
%%%%% new frame %%%%%%%%%%%%%%%%%%%%%%%%%%%%%%%%%%%%%%%%%%%%%%%%%%%%%%%
 \begingroup
  \setbeamertemplate{tuc2 headline 2}{Super Mario}

\begin{frame}
  \textbf{Super Mario}
   
   \begin{itemize}
    % primär: Demonstration, dass Framework flexibel ist, abseits von (wissenschaftlichen) Rechnen
   
    \item Absolvieren des ersten Levels: Sequenz von Tasten-Eingaben (Konfig.)
    % erklären, dass das konfig ist 
    
    % Bild ?
    
    \item Tasten-Eingaben $\rightarrow$ Speichern in Datei $\rightarrow$ Abspielen in NES-Emulator
    % Bsp. ggf. zeigen
    
    \item Wiedergabe in Emulator schneller als in Echtzeit
    \item Auswerten mehrerer Instanzen des Emulators parallel
    % OpenTuner: Parameter --parallelism -> geht hier so einfach da Ergebnisse sich nicht direkt beeinflussen
    
    \text{}
    
    \item Ziel: Maximieren der \#Pixel die Mario nach rechts läuft
    % ! WICHTIG ! : game grafik wird NICHT ausgewertet
    
    \item Suchraum: horizontale Bewegung nach rechts
    % sonst random eingaben, mario bewegt sich nicht zielstrebig
    
    \item Eingaben-Modellierung:
      \begin{itemize}
        \item welche Taste
        \item zu welchem Zeitpunkt (Frame)
        \item wie lange gedrückt wird
      \end{itemize}
    \end{itemize}
   \end{frame}
\endgroup

%%%%% new frame %%%%%%%%%%%%%%%%%%%%%%%%%%%%%%%%%%%%%%%%%%%%%%%%%%%%%%%
 \begingroup
  \setbeamertemplate{tuc2 headline 2}{Super Mario}
  \begin{frame}
  
  \begin{figure}
    \centering\includegraphics[width=\textwidth]{bilder/mario-tuning2}
    \caption{Tuning-Prozess}
  \end{figure} 
  \end{frame}
  \endgroup
  
%%%%% new frame %%%%%%%%%%%%%%%%%%%%%%%%%%%%%%%%%%%%%%%%%%%%%%%%%%%%%%%
 \begingroup
  \setbeamertemplate{tuc2 headline 2}{Super Mario}
  \begin{frame}
  
    \begin{figure}
    \centering\includegraphics[width=0.7\textwidth]{bilder/mario-won2}
    \caption{Ziel erreicht}
  \end{figure}
  
  \centering\fbox{\parbox[c]{4.65cm}{Demo: Erstes Level in Super Mario}}
  
  \end{frame}  
  \endgroup
  
%%%%% new frame %%%%%%%%%%%%%%%%%%%%%%%%%%%%%%%%%%%%%%%%%%%%%%%%%%%%%%%
  \section{Aktueller Stand}
  
    \begin{frame}
      \tableofcontents[currentsection]
    \end{frame}
    
%%%%% new frame %%%%%%%%%%%%%%%%%%%%%%%%%%%%%%%%%%%%%%%%%%%%%%%%%%%%%%%

  \begingroup
%   \setbeamertemplate{tuc2 headline 2}{Super Mario}
  \begin{frame}
  
  \begin{itemize}
    \item \url{www.github.com/jansel/opentuner}
    \item MIT-Lizenz
    \item Entwicklung 2014-2015
    \item Erweiterung durch Nutzer
    \item gute Dokumentation
    \item Ziel: Einsatz in bestehenden und neuen Domänen
  \end{itemize}
  
  
  
  \end{frame}  
  \endgroup
    
%%%%% Section %%%%%%%%%%%%%%%%%%%%%%%%%%%%%%%%%%%%%%%%%%%%%%%%%%%%%%%%%
  \section{Zusammenfassung}
  
    \begin{frame}
      \tableofcontents[currentsection]
    \end{frame}
    
%%%%% new frame %%%%%%%%%%%%%%%%%%%%%%%%%%%%%%%%%%%%%%%%%%%%%%%%%%%%%%%
  
  \begin{frame}
  
   \textbf{PRO\\}
   \begin{itemize}
      \item Flexibilität und Erweiterbarkeit
      % flexibel: untersch. Bsp erwähnen

      \item einfache und klare Handhabung
      % Programmierer "stellt ein" anstatt Hardliner mäßig zu coden 
      % (keineMachine-Learning Expertise nötig)
      
      \item viele Beispielanwendungen
      
      \item mehrere Optimierungsziele (Kombination)
      % lässt trotzdem erfahrenen Nutzern viel Möglichkeiten auf tieferer Ebene Framework 
      % nach Belieben zu nutzen/erweitern

      \item bessere Ergebnisse als bisherige Autotuner
      % HPL, Halide, ...
        
     \end{itemize}

    \text{} % fake leerzeile
      
   \textbf{CONTRA\\}
    \begin{itemize}
      \item Optimierungsgrad variiert stark
      % gute Modellierung, Komplexität des Problems
      
      \item Trade-off: Zeitaufwand $\leftrightarrow$ Nutzen
      \end{itemize}
    
  \end{frame}
  
  
%%%%% Section %%%%%%%%%%%%%%%%%%%%%%%%%%%%%%%%%%%%%%%%%%%%%%%%%%%%%%%%%
\begingroup
\setbeamertemplate{tuc2 headline 1}{Quellen}

\begin{frame}

  \textbf{Quellen \newline}
      
  \begin{thebibliography}{9}

  %\bibitem{lamport94} Leslie Lamport, \emph{\LaTeX: a document preparation system},
   % Addison Wesley, Massachusetts, 2nd edition, 1994.
   
  %%%%%%%%%%%%%%%%%%%
  %%%%%%%%%%%%%%%%%%%
  %  TODO: Zitatstile -> Reihenfolge: Autor, Titel, Jahr, Verlag, Seite, …
  %%%%%%%%%%%%%%%%%%%
  %%%%%%%%%%%%%%%%%%%
   
  \bibitem{img-OT-paper} Jason Ansel, Shoaib Kamil, Kalyan Veeramachaneni, Jonathan Ragan-Kelley, Jeffrey Bosboom, Una-May O'Reilly, Saman Amarasinghe. \emph{OpenTuner: An Extensible Framework for Program Autotuning.}
    2014.
    
  
  \bibitem{handsOn-pdf} Shoaib Kamil. \emph{Hands on session.} Programming Language Design and Implementation 2015
  
  \bibitem{opentuner-git} \url{www.github.com/jansel/opentuner} \footnotesize (08.01.2017)
  \normalsize
  \bibitem{opentuner-site} \url{www.opentuner.org} \footnotesize (05.12.2016)
  

  \end{thebibliography}

\end{frame}
\endgroup

\end{document}
