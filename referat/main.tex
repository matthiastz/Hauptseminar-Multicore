%%%%%%%%%%%%%%%%%%%%%%%%%%%%%%%%%%%%%%%%%%%%%%%%%%%%%%%%%%%%%%%%%%%%%%%%%%%%%%%
%                                                                             %
% Anwendungsbeispiel für das Beamer-Template TU Chemnitz                      %
% (c) Mario Haustein (mario.haustein@hrz.tu-chemnitz.de), 2013-2014           %
%                                                                             %
%%%%%%%%%%%%%%%%%%%%%%%%%%%%%%%%%%%%%%%%%%%%%%%%%%%%%%%%%%%%%%%%%%%%%%%%%%%%%%%

\usepackage[utf8]{inputenc}
\usepackage{babel}
\usepackage{floatflt}
 \usepackage{float}
\usepackage{graphics}

% TUC-Templates laden.
\usetheme[fakcolor=if]{tuc2014}
\mode<article>{\usepackage{beamerarticletuc2014}}

%
% Weitere Anpassungen nach Bedarf.
%


% Metadaten
\title{Professur Praktische Informatik}
\subtitle{OpenTuner: An Extensible Framework for Program Autotuning}
%\subtitle{Hauptseminar}
\author{Matthias Tietz\\Betreuer: Dr. Michael Hofmann}

% Betreuer: Dr. Michael Hofmann

\date{\today}
\institute[TUC]

\begin{document}
    \tucthreeheadlines
    \begin{frame}
      \titlepage
    \end{frame}

    \tucnarrowframe
    \begingroup
% kein logo
%    \logo{\includegraphics[width=\hsize]{Grafiken/logo}}

    \begin{frame}
      \frametitle{Gliederung}
      \tableofcontents
    \end{frame}

    \endgroup
    \tucwideframe
    \tuctwoheadlines

    %
    % Inhalt
    %
    
%%%%% section %%%%%%%%%%%%%%%%%%%%%%%%%%%%%%%%%%%%%%%%%%%%%%%%%%%%%%%%%

    \section{Einleitung}
    
    \subsection{Problemstellung}

%%%%% new frame %%%%%%%%%%%%%%%%%%%%%%%%%%%%%%%%%%%%%%%%%%%%%%%%%%%%%%%   

    \begin{frame}
    
    \begin{itemize}
      \item Suchraum: Menge von Parametern die durchsucht werden soll
      \item geeignete Suchverfahren abhängig von der Beschaffenheit dieser Menge
      \item komplexe Struktur und Größe des Suchraums macht Handoptimierung oder vollständige Suche
      unmöglich (bzw. extrem ineffizient) \\ $\rightarrow$ Nadel im Heuhaufen
    \end{itemize}
      
    \begin{figure}[ht]
      \centering	      
      \includegraphics[scale=0.2]{bilder/3dpolyp}
      \label{3dpoly}
    \end{figure}
    
    \begin{itemize}
      \item Ziele:
      \begin{itemize}
        \item automatisierter und einfacher Optimierungsprozess
        \item bessere und portierbare Performance von domänenspezifischen Programmen
      \end{itemize}
    \end{itemize}

    \end{frame}
    
    \subsection{Warum OpenTuner?}
    
%%%%% new frame %%%%%%%%%%%%%%%%%%%%%%%%%%%%%%%%%%%%%%%%%%%%%%%%%%%%%%%   
      
    \begin{frame}
    \textbf{Die 3 wesentlichen Anforderungen an ein Autotuning-Framework:}
    
    \begin{itemize}
       
    \item 1. Eine passende Konfigurations-Repräsentation
        \begin{itemize}
          \item Darstellung der domänenspezif. Datenstrukturen und Bedingungen
          \item Qualität dieser Repräsentation entscheidend für Effizienz des Autotuners \newline
        \end{itemize}
        
    \item 2. Größe des validen Konfigurations-Raumes
        \begin{itemize}
          \item durch Kürzen des Konfigurations-Raumes geht für viele Probleme gute Lösungen verloren
          (bei bisherigen Autotunern ist dies gängige Praxis, da vollständige Suche)
          \item rießige Konfigurationsräume möglich $\rightarrow$  intelligente Suchtechniken notwendig \newline
        \end{itemize}
        
    \item 3. Landschaft des Konfigurations-Raumes
      \begin{itemize}
        \item Suchräume in der Praxis meist sehr komplex 
        \item domänenspezif. Suchtechniken notwendig um optimale Lösung effizient zu ermitteln
      \end{itemize}
               
    \end{itemize}
    \end{frame}
    
%%%%% new frame %%%%%%%%%%%%%%%%%%%%%%%%%%%%%%%%%%%%%%%%%%%%%%%%%%%%%%%   

    \begin{frame}
    \textbf{Deshalb OpenTuner:}
    
    \begin{itemize}
      \item Erstellen domänenspezifischer und multi-objective Programm-Autotuner
      \item vollständig anpassbare Konfigurations-Repräsentation
      \item erweiterbare Repräsentation für Suchtechniken und Datentypen
      \item Kombination mehrerer Suchtechniken (\textit{Ensembles}), dynamische Zuweisung der
      Testanteile für die jeweiligen Suchtechniken
      \item einfache Schnittstelle zur Kommunikation mit dem zu optimierenden Programm           
    \end{itemize}
    \end{frame}

%%%%% new frame %%%%%%%%%%%%%%%%%%%%%%%%%%%%%%%%%%%%%%%%%%%%%%%%%%%%%%%   
    
  
%%%%% section %%%%%%%%%%%%%%%%%%%%%%%%%%%%%%%%%%%%%%%%%%%%%%%%%%%%%%%%%
    %\section{Anwendungsgebiete}
    
%%%%% new frame %%%%%%%%%%%%%%%%%%%%%%%%%%%%%%%%%%%%%%%%%%%%%%%%%%%%%%%   
%%%%% section %%%%%%%%%%%%%%%%%%%%%%%%%%%%%%%%%%%%%%%%%%%%%%%%%%%%%%%%%
  \section{Das OpenTuner Framework}   
  \subsection{Allgemeines}
    
  \begin{frame}
  \textbf{}
    
  \begin{itemize}
    \item Autotuning-Problem $\rightarrow$ Suchproblem
    \item Suchraum: Menge der Konfigurationen (Belegung von Parametern)
    \item Messung: 1 konkrete Konfig. wird gemessen: Ausführung $\rightarrow$ Ergebnis
    \item Möglichkeit mehrere Messungen parallel auszuführen
    
    \begin{figure}[ht]
      \centering	      
      \includegraphics[scale=0.75]{bilder/smdb}
      \label{smdb}
    \end{figure}

  \end{itemize}
  \end{frame}
    
%%%%% new frame %%%%%%%%%%%%%%%%%%%%%%%%%%%%%%%%%%%%%%%%%%%%%%%%%%%%%%%
  \subsection{Verwendung}

  \begin{frame}
  \textbf{Verwendung}
  \begin{itemize}
    \item 1. Suchraum definieren (Konfig.-Manipulator)
    \item 2. \texttt{run()}-Methode definieren: Auswerten der Konfig. im Suchraum $\rightarrow$ Ergebnis
    \item 3. Festlegen des Optimierungsziels (Zeit, Energie, Genauigkeit, Kombination…)
    \item Umsetzung mittels kleinem Python-Programm (OpenTuner API)
  \end{itemize}
  \textbf{Suchtechniken}
  \begin{itemize}
    \item OpenTuner stellt Suchtechniken für viele Suchraum-Typen bereit
    \item Ausführen mehrerer Suchtechniken gleichzeitig (Ensembles)
    \item dynamische Testzuweisung anhand Erfolges dieser Techniken
    \item erweiterbar: benutzerdefinierte Suchtechniken
  \end{itemize}
  \end{frame}
  
%%%%% new frame %%%%%%%%%%%%%%%%%%%%%%%%%%%%%%%%%%%%%%%%%%%%%%%%%%%%%%%

  \subsection{Konfigurations-Manipulator}

  \begin{frame}
  \textbf{Konfigurations-Manipulator}
  \begin{itemize}
    \item Abstraktionsschicht zwischen Suchtechnik und roher Konfigurations-Struktur
    \item Liste der Parameter/Datenstruktur ist dynamisch erweiterbar
    \item Konfiguration wird als Dictionary verwaltet \newline
  \end{itemize}
  \textbf{Parameter-Typen}
  \begin{itemize}
    \item jeder Parametertyp ist verantwortlich für Schnittstelle zwischen roher 
    Parameterrepräsentation und stand. Ansicht dieses Parameters für die Suchtechnik
    \item Parameterrepräsentation und Abstraktion erweiterbar/konfigurierbar
  \end{itemize}
  \end{frame}
  
%%%%% new frame %%%%%%%%%%%%%%%%%%%%%%%%%%%%%%%%%%%%%%%%%%%%%%%%%%%%%%%
  \begin{frame}
      \begin{figure}[ht]
      \centering	      
      \includegraphics[scale=1]{bilder/paramtypes}
      \label{paramtypes}
    \end{figure}
    
  \textbf{Primitive Parameter}
  \begin{itemize}
    \item numerische Werte mit Unter-/Obergrenze
    \item \texttt{Float} und \texttt{LogFloat (-Int)} gleiche Repräsentation in der Konfiguration, aber 
    untersch. Ansicht des zugrundeliegenden Wertes für die Suchtechnik (skaliert)
    \item Grund: ohne Logskal. würde Effekt der Wertänderung mit steigender Parametergröße sinken
    \item ähnlich bei \texttt{PowerOfTwo} $\rightarrow$ Quadrat nur zulässiger Wert des Parameters
  \end{itemize}
  \end{frame}
  
%%%%% new frame %%%%%%%%%%%%%%%%%%%%%%%%%%%%%%%%%%%%%%%%%%%%%%%%%%%%%%%
  \begin{frame}
    \textbf{Komplexe Parameter}
    \begin{itemize}
      \item haben ein variables Set an Manipulatoren, welche stochastische Änderungen an den Parametern 
      vornehmen
      \item einfach erweiterbar, um domänenspezif. Strukturen zum Suchraum hinzuzufügen
      \item \texttt{Boolean, Switch} und \texttt{Enum} bewusst als komplex. Parameter, da Suchtechniken
      bei primitiven Parametern nach Gradients (Steigungen) suchen. Diese Parameter sind aber ungeordnete
      Sammlung $\rightarrow$ es existiert dafür kein Gradient.
      \item \texttt{Permutation} : Liste von Werte inkl. Manipulatoren zur Änderung der Reihenfolge
      \item \texttt{Schedule} : Sonderfall von \texttt{Permutation} topolog. Sortierung nach jeder Änderung
      \item \texttt{Selector} : Mapping von Integer-Input auf Enum-Type (Darstellung als Baum)
    \end{itemize}
    
    \textbf{Parameter-Interaktion}
    \begin{itemize}
      \item Zusätzlich existieren erweiterbare Methoden für die Suchtechniken um zwischen mehreren Parametern 
    zu interagieren. (z.B. Differenz-Funktion)
    \end{itemize}
  \end{frame}
    

    
    %%% Section %%%
    \section{Konkrete Anwendungen der Entwickler}
    
    %%% Section %%%
    \section{Eigene Anwendung}
    
    %%% Section %%%
    \section{Erkenntnisse und Schlussfolgerungen}
    
    %%% Section %%%
    \section{Quellen}
    
    
    
    

\end{document}
