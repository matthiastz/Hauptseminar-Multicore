%%%%%%%%%%%%%%%%%%%%%%%%%%%%%%%%%%%%%%%%%%%%%%%%%%%%%%%%%%%%%%%%%%%%%%%%%%%%%%%
%                                                                             %
% Anwendungsbeispiel für das Beamer-Template TU Chemnitz                      %
% (c) Mario Haustein (mario.haustein@hrz.tu-chemnitz.de), 2013-2014           %
%                                                                             %
%%%%%%%%%%%%%%%%%%%%%%%%%%%%%%%%%%%%%%%%%%%%%%%%%%%%%%%%%%%%%%%%%%%%%%%%%%%%%%%

\usepackage[utf8]{inputenc}
\usepackage{babel}
\usepackage{floatflt}
\usepackage{float}
\usepackage{graphics}
\usepackage{listings}
\usepackage{color}

% TUC-Templates laden.
\usetheme[fakcolor=if]{tuc2014}
\mode<article>{\usepackage{beamerarticletuc2014}}

%
% Weitere Anpassungen nach Bedarf.
%

\definecolor{mygreen}{rgb}{0,0.6,0}
\definecolor{mygray}{rgb}{0.5,0.5,0.5}
\definecolor{mymauve}{rgb}{0.58,0,0.82}

\lstset{ %
  basicstyle=\footnotesize,        % the size of the fonts that are used for the code
  breakatwhitespace=false,         % sets if automatic breaks should only happen at whitespace
  breaklines=true,                 % sets automatic line breaking
  captionpos=b,                    % sets the caption-position to bottom
  commentstyle=\color{mygreen},    % comment style
  deletekeywords={...},            % if you want to delete keywords from the given language
  escapeinside={\%*}{*)},          % if you want to add LaTeX within your code
  frame=single,	                   % adds a frame around the code
  keepspaces=true,                 % keeps spaces in text, useful for keeping indentation of code (possibly needs columns=flexible)
  keywordstyle=\color{blue},       % keyword style
  language=Octave,                 % the language of the code
  morekeywords={*,...},           % if you want to add more keywords to the set
  numbers=left,                    % where to put the line-numbers; possible values are (none, left, right)
  numbersep=5pt,                   % how far the line-numbers are from the code
  numberstyle=\tiny\color{mygray}, % the style that is used for the line-numbers
  rulecolor=\color{black},         % if not set, the frame-color may be changed on line-breaks within not-black text (e.g. comments (green here))
  showspaces=false,                % show spaces everywhere adding particular underscores; it overrides 'showstringspaces'
  showstringspaces=false,          % underline spaces within strings only
  showtabs=false,                  % show tabs within strings adding particular underscores
  stepnumber=2,                    % the step between two line-numbers. If it's 1, each line will be numbered
  stringstyle=\color{mymauve},     % string literal style
  tabsize=2,	                   % sets default tabsize to 2 spaces
  title=\lstname                   % show the filename of files included with \lstinputlisting; also try caption instead of title
}


% Metadaten
\title{Professur Praktische Informatik}
\subtitle{OpenTuner: An Extensible Framework for Program Autotuning}
%\subtitle{Hauptseminar}
\author{Matthias Tietz\\Betreuung: Prof. Dr. Gudula Rünger, Dr. Michael Hofmann}

% Betreuer: Prof. Dr. Gudula Rünger, Dr. Michael Hofmann

\date{\today}
\institute[TUC]

\begin{document}
    \tucthreeheadlines
    \begin{frame}
      \titlepage
    \end{frame}

    \tucnarrowframe
    \begingroup
% kein logo
%    \logo{\includegraphics[width=\hsize]{Grafiken/logo}}

    \begin{frame}
      \frametitle{Gliederung}
      \tableofcontents
    \end{frame}

    \endgroup
    \tucwideframe
    \tuctwoheadlines

    %
    % Inhalt
    %
    
%%%%% section %%%%%%%%%%%%%%%%%%%%%%%%%%%%%%%%%%%%%%%%%%%%%%%%%%%%%%%%%

    \section{Einleitung}
    
%    \subsection{Problemstellung}

%%%%% new frame %%%%%%%%%%%%%%%%%%%%%%%%%%%%%%%%%%%%%%%%%%%%%%%%%%%%%%%   

    \begin{frame}
    \underline{1. Einleitung}
        
    \begin{itemize}
      \item Suchraum: Menge von Parametern die durchsucht werden soll
      \item geeignete Suchverfahren abhängig von der Beschaffenheit dieser Menge
      \item komplexe Struktur und Größe des Suchraums macht Handoptimierung oder vollständige Suche
      unmöglich (bzw. extrem ineffizient) \\ $\rightarrow$ Nadel im Heuhaufen
    \end{itemize}
      
%      \begin{figure}
%      \centering	      
%        \includegraphics[scale=0.2]{bilder/3dpolyp}
%        \label{3dpoly}
%      \end{figure}

%    \hspace{1cm}

  

%     \includegraphics[scale=0.2]{bilder/3dpolyp}
% %    \hspace{1cm}
% 
%     \framebox{  
%       Ziele:
%       Automatisierter und einfacher Optimierungsprozess. 
%       Bessere und portierbare Performance von domänenspezifischen Programmen

%     }  

  \hspace{0.33cm}    
  \includegraphics[width=4cm]{bilder/3dpolyp}
  \hspace{0.33cm}    
  \fbox{\parbox[b][5em][c]{0.55\textwidth}{\textbf{Ziele:}
      Automatisierter und einfacher Optimierungsprozess.
      Bessere und portierbare Performance von domänenspezifischen Programmen.}}

     



    
    

    \end{frame}
    
%   \subsection{Warum OpenTuner?}
    
%%%%% new frame %%%%%%%%%%%%%%%%%%%%%%%%%%%%%%%%%%%%%%%%%%%%%%%%%%%%%%%   
      
    \begin{frame}

    \begin{itemize}
      \item Ziele:
      \begin{itemize}
        \item automatisierter und einfacher Optimierungsprozess
        \item bessere und portierbare Performance von domänenspezifischen Programmen
      \end{itemize}
    \end{itemize}
    
    \textbf{Die 3 wesentlichen Anforderungen an ein Autotuning-Framework:}
    
    \begin{itemize}
       
    \item 1. Eine passende Konfigurations-Repräsentation
        \begin{itemize}
          \item Darstellung der domänenspezif. Datenstrukturen und Bedingungen
          \item Qualität der Repräsentation entscheidend für Effizienz des Autotuners \newline
        \end{itemize}
        
    \item 2. Größe des validen Konfigurations-Raumes
        \begin{itemize}
          \item Kürzen des Konfigurations-Raumes $\rightarrow$ Verlieren guter Lösungen
          (bei bisherigen Autotunern gängige Praxis, da vollständige Suche)
          \item rießige Konfigurationsräume möglich $\rightarrow$  intelligente Suchtechniken notwendig \newline
        \end{itemize}
        
    \item 3. Landschaft des Konfigurations-Raumes
      \begin{itemize}
        \item Suchräume in der Praxis meist sehr komplex 
        \item domänenspezif. Suchtechniken notwendig um optimale Lösung effizient zu ermitteln
      \end{itemize}
    \end{itemize}
        
    \end{frame}
    
%%%%% new frame %%%%%%%%%%%%%%%%%%%%%%%%%%%%%%%%%%%%%%%%%%%%%%%%%%%%%%%   

    \begin{frame}
    
    \textbf{Deshalb OpenTuner:}   
    \begin{itemize}
      \item Erstellen domänenspezifischer und multi-objective Programm-Autotuner
      \item vollständig anpassbare Konfigurations-Repräsentation
      \item erweiterbare Repräsentation für Suchtechniken und Datentypen
      \item Kombination mehrerer Suchtechniken (\textit{Ensembles}), dynamische Zuweisung der
      Testanteile für die jeweiligen Suchtechniken
      \item einfache Schnittstelle zur Kommunikation mit dem zu optimierenden Programm           
    \end{itemize}
    \end{frame}

%%%%% new frame %%%%%%%%%%%%%%%%%%%%%%%%%%%%%%%%%%%%%%%%%%%%%%%%%%%%%%%   
    
  
%%%%% section %%%%%%%%%%%%%%%%%%%%%%%%%%%%%%%%%%%%%%%%%%%%%%%%%%%%%%%%%
    %\section{Anwendungsgebiete}
    
%%%%% new frame %%%%%%%%%%%%%%%%%%%%%%%%%%%%%%%%%%%%%%%%%%%%%%%%%%%%%%%   
%%%%% section %%%%%%%%%%%%%%%%%%%%%%%%%%%%%%%%%%%%%%%%%%%%%%%%%%%%%%%%%
  \section{Das OpenTuner Framework}   
%  \subsection{Allgemeines}
   
  
  \begin{frame}
  
  \underline{2. Das OpenTuner Framework}
  \begin{itemize}
    \item Autotuning-Problem $\rightarrow$ Suchproblem
    \item Suchraum: Menge der Konfigurationen (Belegung von Parametern)
    \item Messung: 1 konkrete Konfig. wird gemessen: Ausführung $\rightarrow$ Ergebnis
    \item Möglichkeit mehrere Messungen parallel auszuführen
    
    \begin{figure}[ht]
      \centering	      
      \includegraphics[scale=0.75]{bilder/smdb}
      \label{smdb}
    \end{figure}

  \end{itemize}
  \end{frame}
    
%%%%% new frame %%%%%%%%%%%%%%%%%%%%%%%%%%%%%%%%%%%%%%%%%%%%%%%%%%%%%%%
%  \subsection{Verwendung}

  \begin{frame}
  \textbf{Verwendung}
  \begin{itemize}
    \item 1. Suchraum definieren (Konfig.-Manipulator)
    \item 2. \texttt{run()}-Methode definieren: Auswerten der Konfig. im Suchraum $\rightarrow$ Ergebnis
    \item 3. Festlegen des Optimierungsziels
    \item Umsetzung mittels kleinem Python-Programm (OpenTuner API), Framework ist ausschließlich in 
    Python geschrieben
  \end{itemize}
  \textbf{Suchtechniken}
  \begin{itemize}
    \item OpenTuner stellt Suchtechniken für viele Suchraum-Typen bereit
    \item Ausführen mehrerer Suchtechniken gleichzeitig (Ensembles)
    \item dynamische Testzuweisung anhand Erfolges dieser Techniken
    \item erweiterbar: benutzerdefinierte Suchtechniken
  \end{itemize}
  \end{frame}
  
%%%%% new frame %%%%%%%%%%%%%%%%%%%%%%%%%%%%%%%%%%%%%%%%%%%%%%%%%%%%%%%

%  \subsection{Konfigurations-Manipulator}

  \begin{frame}
  \textbf{Konfigurations-Manipulator}
  \begin{itemize}
    \item Abstraktionsschicht zwischen Suchtechnik und roher Konfigurations-Struktur
    \item Liste der Parameter/Datenstruktur ist dynamisch erweiterbar
    \item Konfiguration wird als Dictionary verwaltet \newline
  \end{itemize}
  \textbf{Parameter-Typen}
  \begin{itemize}
    \item jeder Parametertyp ist verantwortlich für Schnittstelle zwischen roher 
    Parameterrepräsentation und stand. Ansicht dieses Parameters für die Suchtechnik
    \item Parameterrepräsentation und Abstraktion erweiterbar/konfigurierbar
  \end{itemize}
  \end{frame}
  
%%%%% new frame %%%%%%%%%%%%%%%%%%%%%%%%%%%%%%%%%%%%%%%%%%%%%%%%%%%%%%%
  \begin{frame}
    \begin{figure}[ht]
      \centering	      
      \includegraphics[scale=1.1]{bilder/paramtypes}
      \label{paramtypes}
    \end{figure}
    
  \textbf{Primitive Parameter}
  \begin{itemize}
    \item numerische Werte mit Unter-/Obergrenze
    \item \texttt{Float} und \texttt{LogFloat (-Int)} gleiche Repräsentation in der Konfiguration, aber 
    untersch. Ansicht des zugrundeliegenden Wertes für die Suchtechnik (skaliert)
    \item Grund: ohne Logskal. würde Effekt der Wertänderung mit steigender Parametergröße sinken
    \item ähnlich bei \texttt{PowerOfTwo} $\rightarrow$ Quadrat nur zulässiger Wert des Parameters
  \end{itemize}
  \end{frame}
  
%%%%% new frame %%%%%%%%%%%%%%%%%%%%%%%%%%%%%%%%%%%%%%%%%%%%%%%%%%%%%%%
  \begin{frame}
    \textbf{Komplexe Parameter}
    \begin{itemize}
      \item haben variables Set an Manipulatoren $\rightarrow$ stochastische Parameter-Änderungen
      \item einfach domänenspezif. Strukturen zum Suchraum hinzuzufügen
      \item \texttt{Boolean, Switch} und \texttt{Enum} bewusst als komplex. Parameter, Darstellung
      als ungeordnete Sammlung $\rightarrow$ es existiert kein Gradient (wie bei prim. Parm.)
      \item \texttt{Permutation} : Liste von Werte inkl. Manipulatoren zur randomisierten Änderung der Reihenfolge
      \item \texttt{Schedule} ist Sonderfall von \texttt{Permutation} : topolog. Sortierung nach jeder Änderung
      \item \texttt{Selector} : Mapping von Integer-Input auf Enum-Type (Darstellung als Baum)
    \end{itemize}
  \end{frame}
  
%%%%% new frame %%%%%%%%%%%%%%%%%%%%%%%%%%%%%%%%%%%%%%%%%%%%%%%%%%%%%%%
  \begin{frame}
  \textbf{Parameter-Interaktion}
    \begin{itemize}
      \item Zusätzlich existieren erweiterbare Methoden für die Suchtechniken um zwischen mehreren Parametern 
    zu interagieren. (z.B. Differenz-Funktion)
    \end{itemize}
    
    \textbf{\\}
  
  \textbf{Optimierungsziele}
    \begin{itemize}
      \item OpenTuner unterstützt mehrere Ziele, standardmäßig wird nach der \texttt{Zeit} optimiert
      \item \texttt{Genauigkeit, Energie, Größe} oder ein nutzerdef. Ziel
      \item Es können auch mehrere Ziele zugleich verfolgt werden, bspw. \texttt{Genauigkeit einhalten, gleichzeitig
      Zeit minimieren}
    \end{itemize}
  \end{frame}
  
%%%%% new frame %%%%%%%%%%%%%%%%%%%%%%%%%%%%%%%%%%%%%%%%%%%%%%%%%%%%%%%
  \begin{frame}
    \textbf{Suchen und Messen}
    \begin{itemize}
      \item kommunizieren ausschließlich über die Ergebnis-Datenbank
      \item Motivation für die Unterteilung zwischen diesen beiden Modulen:
      \begin{itemize}
        \item Ermöglichen der Parallelität zwischen mehreren Prozessen (Suchen und Messen)
        \item Autotuning während Ausführung der Anwendung oder in Wartezeit (Online/Sideline Learning)
        \item Mess-Modul einfach ersetzbar ohne das gesamte Framework zu modifizieren (Domäne: Embedded/Mobil
        $\rightarrow$ leichtgewichtiges Mess-Modul)
      \end{itemize}
    \end{itemize} 
    
  \textbf{\\}
  
  \textbf{Ergebnis-Datenbank}
    \begin{itemize}
      \item vollfunktionale SQL-Datenbank
      \item alle grundlegenden DB-Typen unterstützt, default: SQLite 
      \item Abfragen und Eintragen der Ergebnisse in einer Vielzahl von Möglichkeiten
      \item nützlich für die Performance-Beobachtung der Suchtechniken
    \end{itemize}
  \end{frame}
  
%%%%% new frame %%%%%%%%%%%%%%%%%%%%%%%%%%%%%%%%%%%%%%%%%%%%%%%%%%%%%%%
%%%%% Section %%%%%%%%%%%%%%%%%%%%%%%%%%%%%%%%%%%%%%%%%%%%%%%%%%%%%%%%%
    \section{Konkrete Anwendungen der Entwickler}
    
    \begin{frame}
    \underline{3. Konkrete Anwendungen der Entwickler} \newline
    
     Vorstellen eines Beispiels aus einer bestimmten Kategorie \newline
    
    \textbf{a) GCC/G++ Flags}
      \begin{itemize}
        \item klassische Parameter-Optimierung
        \item unterstützte Flags: \texttt{g++ --help=optimizers}
        \item Parameter inkl. zulässiger Wertebereiche: \texttt{params.def} (gcc source code)
        \item Implementierung des Autotuners:
        \begin{itemize}
          \item 1. Erstellen des \texttt{configuration manipulator} : Set der Parameter (Suchraum)
          \item 2. Erstellen der \texttt{run}-Funktion
          \item 3. Festlegen des Optimierungsziels
        \end{itemize}
      \end{itemize}
    \end{frame}
        
%%%%% new frame %%%%%%%%%%%%%%%%%%%%%%%%%%%%%%%%%%%%%%%%%%%%%%%%%%%%%%%
  \begin{frame}
    \begin{figure}[ht]
      \centering	      
      \includegraphics[width=\textwidth]{bilder/ot-3-1a}
      \label{gccpy1a}
    \end{figure}
  \end{frame}
  
%%%%% new frame %%%%%%%%%%%%%%%%%%%%%%%%%%%%%%%%%%%%%%%%%%%%%%%%%%%%%%%
  
  \begin{frame}
    \begin{figure}[ht]
      \centering	      
      \includegraphics[width=\textwidth]{bilder/ot-3-1}
      \label{gccpy1}
    \end{figure}
  \end{frame}
  
%%%%% new frame %%%%%%%%%%%%%%%%%%%%%%%%%%%%%%%%%%%%%%%%%%%%%%%%%%%%%%%
  
  \begin{frame}
    \begin{figure}[ht]
      \centering	      
      \includegraphics[scale=0.9]{bilder/ot-3-2}
      \label{gccpy2}
    \end{figure}
  \end{frame}
    
    
%%%%% Section %%%%%%%%%%%%%%%%%%%%%%%%%%%%%%%%%%%%%%%%%%%%%%%%%%%%%%%%%
    \section{Eigene Anwendung}
    
%%%%% Section %%%%%%%%%%%%%%%%%%%%%%%%%%%%%%%%%%%%%%%%%%%%%%%%%%%%%%%%%
    \section{Erkenntnisse und Schlussfolgerungen}
    
%%%%% Section %%%%%%%%%%%%%%%%%%%%%%%%%%%%%%%%%%%%%%%%%%%%%%%%%%%%%%%%%
    \section{Quellen}
    
    
    
    

\end{document}
