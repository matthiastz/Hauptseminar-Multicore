%%%%%%%%%%%%%%%%%%%%%%%%%%%%%%%%%%%%%%%%%%%%%%%%%%%%%%%%%%%%%%%%%%%%%%%%%%%%%%%
%                                                                             %
% Anwendungsbeispiel für das Beamer-Template TU Chemnitz                      %
% (c) Mario Haustein (mario.haustein@hrz.tu-chemnitz.de), 2013-2014           %
%                                                                             %
%%%%%%%%%%%%%%%%%%%%%%%%%%%%%%%%%%%%%%%%%%%%%%%%%%%%%%%%%%%%%%%%%%%%%%%%%%%%%%%

\usepackage[utf8]{inputenc}
\usepackage{babel}
\usepackage{floatflt}
\usepackage{float}
\usepackage{graphics}
\usepackage{listings}
\usepackage{color}

% TUC-Templates laden.
\usetheme[fakcolor=if]{tuc2014}
\mode<article>{\usepackage{beamerarticletuc2014}}

%
% Weitere Anpassungen nach Bedarf.
%

% Metadaten
\title{OpenTuner: An Extensible Framework \\ for Program Autotuning}
\subtitle{Hauptseminar Wintersemester 2016/2017}
\author{Matthias Tietz}

\date{10.02.2017}
\institute[TUC]

\begin{document}

    \tucthreeheadlines
    
    \begin{frame}
      \titlepage
    \end{frame}

    \begin{frame}

       \frametitle{Gliederung}
       \tableofcontents

    \end{frame}


    \tuctwoheadlines

    %
    % Inhalt
    %
    
%%%%% section %%%%%%%%%%%%%%%%%%%%%%%%%%%%%%%%%%%%%%%%%%%%%%%%%%%%%%%%%

    \section{Einleitung}
    
%    \subsection{Problemstellung}

%%%%% new frame %%%%%%%%%%%%%%%%%%%%%%%%%%%%%%%%%%%%%%%%%%%%%%%%%%%%%%%   

    \begin{frame}
    \textbf{1. Einleitung}
        
    \begin{itemize}
      \item Suchraum: Menge von Parametern die durchsucht werden soll
      \item geeignete Suchverfahren abhängig von der Beschaffenheit dieser Menge
      \item komplexe Struktur und Größe des Suchraums macht Handoptimierung oder vollständige Suche
      unmöglich (bzw. extrem ineffizient) \\ $\rightarrow$ Nadel im Heuhaufen
    \end{itemize}
      
  \hspace{0.33cm}    
  \includegraphics[width=4cm]{bilder/3dpolyp}
  \hspace{0.33cm}    
  \fbox{\parbox[b][5em][c]{0.55\textwidth}{\textbf{Ziele:}
      Automatisierter und einfacher Optimierungsprozess.
      Bessere und portierbare Performance von domänenspezifischen Programmen.}}

    \end{frame}
    

    
%%%%% new frame %%%%%%%%%%%%%%%%%%%%%%%%%%%%%%%%%%%%%%%%%%%%%%%%%%%%%%%   
      
    \begin{frame}
    
    \textbf{Die Herausforderungen bei der Entwicklung eins Autotuning-Frameworks:}
    
    \begin{itemize}
       
    \item 1. Eine passende Konfigurations-Repräsentation
        \begin{itemize}
          \item Darstellung der domänenspezif. Datenstrukturen und Bedingungen
          \item Qualität der Repräsentation entscheidend für Effizienz des Autotuners \newline
        \end{itemize}
        
    \item 2. Größe des gültigen Konfigurations-Raumes
        \begin{itemize}
          \item Kürzen des Konfigurations-Raumes $\rightarrow$ Verlieren guter Lösungen
          (bei bisherigen Autotunern gängige Praxis, da vollständige Suche)
          \item rießige Konfigurationsräume möglich $\rightarrow$  intelligente Suchtechniken notwendig \newline
        \end{itemize}
        
    \item 3. Beschaffenheit des Konfigurations-Raumes
      \begin{itemize}
        \item Suchräume in der Praxis meist sehr komplex 
        \item domänenspezif. Suchtechniken notwendig um optimale Lösung effizient zu ermitteln
      \end{itemize}
    \end{itemize}
        
    \end{frame}
    
%%%%% new frame %%%%%%%%%%%%%%%%%%%%%%%%%%%%%%%%%%%%%%%%%%%%%%%%%%%%%%%   

    \begin{frame}
    
    \textbf{Eigenschaften von OpenTuner:}   
    \begin{itemize}
      \item Erstellen domänenspezifischer und multi-objective Programm-Autotuner
      \item vollständig anpassbare Konfigurations-Repräsentation
      \item erweiterbare Repräsentation für Suchtechniken und Datentypen
      \item Kombination mehrerer Suchtechniken (\textit{Ensembles}), dynamische Zuweisung der
      Testanteile für die jeweiligen Suchtechniken
      \item einfache Schnittstelle zur Kommunikation mit dem zu optimierenden Programm           
    \end{itemize}
    \end{frame}

%%%%% new frame %%%%%%%%%%%%%%%%%%%%%%%%%%%%%%%%%%%%%%%%%%%%%%%%%%%%%%%   
    
  
%%%%% section %%%%%%%%%%%%%%%%%%%%%%%%%%%%%%%%%%%%%%%%%%%%%%%%%%%%%%%%%
    %\section{Anwendungsgebiete}
    
%%%%% new frame %%%%%%%%%%%%%%%%%%%%%%%%%%%%%%%%%%%%%%%%%%%%%%%%%%%%%%%   
%%%%% section %%%%%%%%%%%%%%%%%%%%%%%%%%%%%%%%%%%%%%%%%%%%%%%%%%%%%%%%%
  \section{OpenTuner Framework} 
  
  \begin{frame}
  
  \textbf{2. OpenTuner Framework}
  \begin{itemize}
    \item Autotuning-Problem $\rightarrow$ Suchproblem
    \item Suchraum: Menge der Konfigurationen (Belegung von Parametern)
    \item Messung: 1 konkrete Konfig. wird gemessen: Ausführung $\rightarrow$ Ergebnis
    \item Möglichkeit mehrere Messungen parallel auszuführen
    
    \begin{figure}[ht]
      \centering	      
      \includegraphics[scale=0.75]{bilder/smdb}
      \label{smdb}
    \end{figure}

  \end{itemize}
  \end{frame}
    
%%%%% new frame %%%%%%%%%%%%%%%%%%%%%%%%%%%%%%%%%%%%%%%%%%%%%%%%%%%%%%%
%  \subsection{Verwendung}

  \begin{frame}
  \textbf{Verwendung}
  \begin{itemize}
    \item 1. Suchraum definieren (Konfig.-Manipulator)
    \item 2. \texttt{run()}-Methode definieren: Auswerten der Konfig. im Suchraum $\rightarrow$ Ergebnis
    \item 3. Festlegen des Optimierungsziels
    \item Umsetzung mittels kleinem Python-Programm (OpenTuner API), Framework ist ausschließlich in 
    Python geschrieben
  \end{itemize}
  \textbf{Suchtechniken}
  \begin{itemize}
    \item OpenTuner stellt Suchtechniken für viele Suchraum-Typen bereit
    \item Ausführen mehrerer Suchtechniken gleichzeitig (Ensembles)
    \item dynamische Testzuweisung anhand Erfolges dieser Techniken
    \item erweiterbar: benutzerdefinierte Suchtechniken
  \end{itemize}
  \end{frame}
  
%%%%% new frame %%%%%%%%%%%%%%%%%%%%%%%%%%%%%%%%%%%%%%%%%%%%%%%%%%%%%%%

%  \subsection{Konfigurations-Manipulator}

  \begin{frame}
  \textbf{Konfigurations-Manipulator}
  \begin{itemize}
    \item Abstraktionsschicht zwischen Suchtechnik und roher Konfigurations-Struktur
    \item Liste der Parameter/Datenstruktur ist dynamisch erweiterbar
  \end{itemize}
  \textbf{Parameter-Typen}
  \begin{itemize}
    \item jeder Parametertyp ist verantwortlich für Schnittstelle zwischen roher 
    Parameterrepräsentation und stand. Ansicht dieses Parameters für die Suchtechnik
    \item Parameterrepräsentation und Abstraktion erweiterbar/konfigurierbar
  \end{itemize}
  \end{frame}
  
%%%%% new frame %%%%%%%%%%%%%%%%%%%%%%%%%%%%%%%%%%%%%%%%%%%%%%%%%%%%%%%
  \begin{frame}
    \begin{figure}[ht]
      \centering	      
      \includegraphics[scale=1.1]{bilder/paramtypes}
      \label{paramtypes}
    \end{figure}
    
  \textbf{Primitive Parameter}
  \begin{itemize}
    \item numerische Werte mit Unter-/Obergrenze
    \item \texttt{Float} und \texttt{LogFloat (-Int)} gleiche Repräsentation in der Konfiguration, aber 
    untersch. Ansicht des zugrundeliegenden Wertes für die Suchtechnik (skaliert)
    \item Grund: ohne Logskal. würde Effekt der Wertänderung mit steigender Parametergröße sinken
  \end{itemize}
  \end{frame}
  
%%%%% new frame %%%%%%%%%%%%%%%%%%%%%%%%%%%%%%%%%%%%%%%%%%%%%%%%%%%%%%%
  \begin{frame}
    \textbf{Komplexe Parameter}
    \begin{itemize}
      \item haben variables Set an Manipulatoren $\rightarrow$ stochastische Parameter-Änderungen
      \item einfach domänenspezif. Strukturen zum Suchraum hinzuzufügen
      \item \texttt{Boolean, Switch} und \texttt{Enum} bewusst als komplex. Parameter, Darstellung
      als ungeordnete Sammlung $\rightarrow$ es existiert kein Gradient (wie bei prim. Parm.)
      \item \texttt{Permutation} : Liste von Werte inkl. Manipulatoren zur randomisierten Änderung der Reihenfolge
      \item \texttt{Schedule} ist Sonderfall von \texttt{Permutation} : topolog. Sortierung nach jeder Änderung
      \item \texttt{Selector}
    \end{itemize}
  \end{frame}
  
%%%%% new frame %%%%%%%%%%%%%%%%%%%%%%%%%%%%%%%%%%%%%%%%%%%%%%%%%%%%%%%
  \begin{frame}
  \textbf{Parameter-Interaktion}
    \begin{itemize}
      \item Zusätzlich existieren erweiterbare Methoden für die Suchtechniken um zwischen mehreren Parametern 
    zu interagieren. (z.B. Differenz-Funktion)
    \end{itemize}
    
    \textbf{\\}
  
  \textbf{Optimierungsziele}
    \begin{itemize}
      \item OpenTuner unterstützt mehrere Ziele, standardmäßig wird nach der \texttt{Zeit} optimiert
      \item \texttt{Genauigkeit, Energie, Größe} oder ein nutzerdef. Ziel
      \item Es können auch mehrere Ziele zugleich verfolgt werden, bspw. \texttt{Genauigkeit einhalten, gleichzeitig
      Zeit minimieren}
    \end{itemize}
  \end{frame}
    
%%%%% new frame %%%%%%%%%%%%%%%%%%%%%%%%%%%%%%%%%%%%%%%%%%%%%%%%%%%%%%%
%%%%% Section %%%%%%%%%%%%%%%%%%%%%%%%%%%%%%%%%%%%%%%%%%%%%%%%%%%%%%%%%
    \section{Konkrete Anwendungen der Entwickler}
    
    \begin{frame}
    \textbf{3. Existierende Anwendungen der Entwickler} \newline
      
    \textbf{a) GCC/G++ Flags}
      \begin{itemize}
        \item klassische Parameter-Optimierung
        \item unterstützte Flags: \texttt{g++ --help=optimizers}
        \item Parameter inkl. zulässiger Wertebereiche: \texttt{params.def} (gcc source code)
        \item Implementierung des Autotuners:
        \begin{itemize}
          \item 1. Erstellen des \texttt{configuration manipulator} : Menge der Parameter (Suchraum) \newline
          \emph{Optimierungslevel, GXX-Flags/-Parameter}
          \item 2. Erstellen der \texttt{run}-Funktion
          \item 3. Festlegen des Optimierungsziels
        \end{itemize}
      \end{itemize}
    \end{frame}
        
%%%%% new frame %%%%%%%%%%%%%%%%%%%%%%%%%%%%%%%%%%%%%%%%%%%%%%%%%%%%%%%
  \begin{frame}
    \begin{figure}[ht]
      \centering	      
      \includegraphics[width=\textwidth]{bilder/ot-3-1a}
      \label{gccpy1a}
    \end{figure}
  \end{frame}
  
%%%%% new frame %%%%%%%%%%%%%%%%%%%%%%%%%%%%%%%%%%%%%%%%%%%%%%%%%%%%%%%
  
  \begin{frame}
    \begin{figure}[ht]
      \centering	      
      \includegraphics[width=\textwidth]{bilder/ot-3-1}
      \label{gccpy1}
    \end{figure}
  \end{frame}
  
%%%%% new frame %%%%%%%%%%%%%%%%%%%%%%%%%%%%%%%%%%%%%%%%%%%%%%%%%%%%%%%
  
  \begin{frame}
    \begin{figure}[ht]
      \centering	      
      \includegraphics[scale=0.9]{bilder/ot-3-2}
      \label{gccpy2}
    \end{figure}
  \end{frame}
  
%%%%% new frame %%%%%%%%%%%%%%%%%%%%%%%%%%%%%%%%%%%%%%%%%%%%%%%%%%%%%%%
  
 \begin{frame}
   \begin{itemize}
     \item die Methode \texttt{run} implementiert die Messfunktion der Konfigurationen
     \item Konfiguration: spez. GXX-Befehlszeile (Opt, Flags, Params)
     \item Ausführen der Befehlszeile $\rightarrow$ Executable erstellen
     \item \texttt{call\_program} : Ausführen und Messen des Programms
     \item \texttt{Ergebnis} wird erstellt und in Datenbank geschrieben 
     (record type abhängig von Optimierungsziel) \newline
     \item vergleichbare Anwendung: Halide, PetaBricks, Stencil
   \end{itemize}
 \end{frame}
 
%%%%% new frame %%%%%%%%%%%%%%%%%%%%%%%%%%%%%%%%%%%%%%%%%%%%%%%%%%%%%%%
  \begin{frame}
      \begin{figure}[ht]
      \centering	      
      \includegraphics[width=\textwidth]{bilder/outfile_8.pdf}
      \label{gccResults}
    \end{figure}
  \end{frame}    
    
%%%%% new frame %%%%%%%%%%%%%%%%%%%%%%%%%%%%%%%%%%%%%%%%%%%%%%%%%%%%%%%
  \begin{frame}
   \textbf{b) Unitary Matrices}
   
    \begin{itemize}
      \item Synthetisierung von Matrizen in optimaler Zeit
      \item \textbf{bisher} traditioneller Autotuner zur Programmoptimierung
      \item \textbf{hier} Suche als Subroutine zur Laufzeit
      \item das Problem hat fixes Set von Operatoren (Controls), repräsentiert als Matrizen
      \item Ziel: Finden einer Sequenz von Operatoren, sodass Matrixmultiplikation die Zielmatrix ergibt
      \item Zielfunktion: Abstand (Genauigkeitswert) des Produktes der aktuellen Sequenz zum Ziel (trace fidelity)
    \end{itemize}
  \end{frame}
  
%%%%% new frame %%%%%%%%%%%%%%%%%%%%%%%%%%%%%%%%%%%%%%%%%%%%%%%%%%%%%%%
\begin{frame}
    \begin{figure}[ht]
      \centering	      
      \includegraphics[width=0.9\textwidth]{bilder/unitary.pdf}
      \label{gccpy1a}
    \end{figure}
\end{frame}

%%%%% new frame %%%%%%%%%%%%%%%%%%%%%%%%%%%%%%%%%%%%%%%%%%%%%%%%%%%%%%%
\begin{frame}
   \textbf{c) Super Mario}
   
   \begin{itemize}
    \item Abschließen des ersten Levels durch Sequenz von Button-Eingaben 
    \item
   \end{itemize}
   
\end{frame}


  
%%%%% Section %%%%%%%%%%%%%%%%%%%%%%%%%%%%%%%%%%%%%%%%%%%%%%%%%%%%%%%%%
    \section{Anwendungsbeispiele}
    
%%%%% new frame %%%%%%%%%%%%%%%%%%%%%%%%%%%%%%%%%%%%%%%%%%%%%%%%%%%%%%%
    \begin{frame}
    \begin{itemize}
      \item abc abc abc
    \end{itemize}
    \end{frame}
    
%%%%% Section %%%%%%%%%%%%%%%%%%%%%%%%%%%%%%%%%%%%%%%%%%%%%%%%%%%%%%%%%
    \section{Zusammenfassung}
     \begin{itemize}
      \item abc abc abc
    \end{itemize}
    
%%%%% Section %%%%%%%%%%%%%%%%%%%%%%%%%%%%%%%%%%%%%%%%%%%%%%%%%%%%%%%%%
    \section{Quellen}
    
    
    
    

\end{document}
