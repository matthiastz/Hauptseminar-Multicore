%%%%%%%%%%%%%%%%%%%%%%%%%%%%%%%%%%%%%%%%%%%%%%%%%%%%%%%%%%%%%%%%%%%%%%%%%%%%%%%
%                                                                             %
% Anwendungsbeispiel für das Beamer-Template TU Chemnitz                      %
% (c) Mario Haustein (mario.haustein@hrz.tu-chemnitz.de), 2013-2014           %
%                                                                             %
%%%%%%%%%%%%%%%%%%%%%%%%%%%%%%%%%%%%%%%%%%%%%%%%%%%%%%%%%%%%%%%%%%%%%%%%%%%%%%%

\usepackage[utf8]{inputenc}
\usepackage{babel}
\usepackage{floatflt}
 \usepackage{float}
\usepackage{graphics}

% TUC-Templates laden.
\usetheme[fakcolor=if]{tuc2014}
\mode<article>{\usepackage{beamerarticletuc2014}}

%
% Weitere Anpassungen nach Bedarf.
%


% Metadaten
\title{Professur Praktische Informatik}
\subtitle{OpenTuner: An Extensible Framework for Program Autotuning}
%\subtitle{Hauptseminar}
\author{Matthias Tietz\\Betreuer: Dr. Michael Hofmann}

% Betreuer: Dr. Michael Hofmann

\date{\today}
\institute[TUC]

\begin{document}
    \tucthreeheadlines
    \begin{frame}
      \titlepage
    \end{frame}

    \tucnarrowframe
    \begingroup
% kein logo
%    \logo{\includegraphics[width=\hsize]{Grafiken/logo}}

    \begin{frame}
      \frametitle{Gliederung}
      \tableofcontents
    \end{frame}

    \endgroup
    \tucwideframe
    \tuctwoheadlines

    %
    % Inhalt
    %
    
    %%% Section %%%

    \section{Einleitung}
    
    \subsection{Problemstellung}
      
    \begin{frame}
    
    \begin{itemize}
      \item Suchraum: Menge von Parametern die durchsucht werden soll
      \item geeignete Suchverfahren abhängig von der Beschaffenheit dieser Menge
      \item komplexe Struktur und Größe des Suchraums macht Handoptimierung oder vollständige Suche
      unmöglich (bzw. extrem ineffizient) \\ $\rightarrow$ Nadel im Heuhaufen
    \end{itemize}
      
    \begin{figure}[ht]
      \centering	      
      \includegraphics[scale=0.2]{bilder/3dpolyp}
      \label{3dpoly}
    \end{figure}
    
    \begin{itemize}
      \item Ziele:
      \begin{itemize}
        \item automatisierter und einfacher Optimierungsprozess
        \item bessere und portierbare Performance von domänenspezifischen Programmen
      \end{itemize}
    \end{itemize}

    \end{frame}
    
    \subsection{Warum OpenTuner?}
    
    \begin{frame}
    \textbf{Warum OpenTuner?}
    
    \begin{itemize}
    \normalsize 
      \item Erstellen domänenspezifischer und multi-objective Programm-Autotuner
      \item vollständig anpassbare Konfigurations-Repräsentation
      \item erweiterbare Repräsentation für Suchtechniken und Datentypen
      $\rightarrow$ essentiell für (gute) Ergebnisse
      \item einfache Schnittstelle zur Kommunikation mit dem zu optimierenden Programm
      \item Kombination mehrerer Suchtechniken (\textit{Ensembles}), dynamische Zuweisung der
      Testanteile für die jeweiligen Suchtechniken
      \item Open-Source-Framework
           
    \end{itemize}
    \end{frame}
    
    %%% Section %%%
    \section{Anwendungsgebiete}
    
    %%% Section %%%
    \section{Das OpenTuner Framework}
    
    %%% Section %%%
    \section{Konkrete Anwendungen der Entwickler}
    
    %%% Section %%%
    \section{Eigene Anwendung}
    
    %%% Section %%%
    \section{Erkenntnisse und Schlussfolgerungen}
    
    %%% Section %%%
    \section{Quellen}
    
    
    
    

\end{document}
